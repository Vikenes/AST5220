\section{Milestone I}\label{M1}
In this section we will examine the evolution of the Universe's uniform background. Our primary objective is to develop methods for computing the Hubble parameter and related time- and distance measures. These methods provide a first step towards further investigations and modelling of the early Universe. To compute the background cosmology, we will solve ordinary differential equations (ODEs) numerically, using cosmological parameters obtained from the Planck Collaboration \cite*{Planck2020}. The parameters we will use are listed in Eq. \eqref{eq:Appendix:Fiducial_cosmology_parameters} in  \appref{app:M1:fiducial_parameters}. One crucial aspect in the process is validating our model. We will therefore develop some simple methods for comparing our result. This will mainly involve considering simplified cases where analytical solutions can be obtained. 

Our primary focus in this section concerns methods where the cosmological parameters are given from the start. Another interesting aspect is to use data to constrain cosmological parameters. To do this, we will use data from supernova observations \cite{Supernova2014Betoule}, containing luminosity distance associated with different values of redshift. By employing the numerical methods we develop initially, we will try to estimate optimal values of three cosmological parameters, by implement a simple Markov chain Monte Carlo (MCMC) algorithm. The parameters we will be sampling are $h,\,\omn$ and $\okn$. From these results, we will investigate confidence regions of $\omn$ and $\oln$, and try to estimate a probability distribution function (PDF) for the Hubble parameter. 

The code for this milestone can be found on my GitHub repository: \url{https://github.com/Vikenes/AST5220/tree/main/projects/milestone1}



\subsection{Theory}\label{ssec:M1:theory}

\subsubsection{Density parameters and Hubble factor}

The Friedmann equation can be written in terms of density parameters, $\Omega_i\equiv \rho_i/\rho_c$, where $\rho_c\equiv 3H^2/8\pi G$ is the critical density. The density of a given species, $i$, evolves as \cite[Eq. (2.61)]{Dodelson}
\begin{equation}
    \rho_i(t)\propto a(t)^{-3(1+w_i)}, \label{eq:M1:theory:rho_i_eos_dependence}
\end{equation}
where we have assumed that the equation of state (EoS) parameter, $w_i \equiv P_i/\rho_i$ \cite[Eq. 2.60]{Dodelson}, is constant. $P_i$ denotes the pressure of the species. 
%
We will limit ourselves to consider three types of species in this report: matter, radiation and dark energy. We will only consider baryons and cold dark matter (CDM) for the matter component, which we express as $\omn=\obn+\ocdmn$. The subscript $0$ is used to refer to today's value. The radiation component we consider is $\oradn=\ogn+\onun$, corresponding to photons and neutrinos, respectively. For the dark energy, we only have the cosmological constant $\oln$.   

Matter, radiation and dark energy have densities evolving according to Eq. \eqref{eq:M1:theory:rho_i_eos_dependence} with $w_i=0,\,1/3$ and $-1$, respectively. Neutrinos having $w=1/3$ only holds since we will assume that the neutrinos are massless. Curvature can be described by $w_i=-1/3$, with $\okn\equiv -kc^2/H_0^2$. The parameter $k$ represents the curvature of the Universe, where $k=0$ corresponds to a flat Universe. With these parameters, the Friedmann Equation can be written as \cite[Eq. (3.14)]{Dodelson}  
\begin{equation}
    H = H_0 \sqrt{\omn a^{-3} + \oradn a^{-4} + \okn a^{-2} + \oln}, \label{eq:M1:theory:Friedmann_H_omegas}
\end{equation}
where $H\equiv\dot{a}/a$ is the Hubble parameter, with the dot denoting a derivative with respect to cosmic time, $t$. For the radiation, $\ogn$ and $\onun$ follow from the temperature of the CMB today, $\tcmb$, and the effective number of massless neutrinos, $N_\mathrm{eff}$. They are given by 
\begin{align}
    \ogn &= 2 \cdot \frac{\pi^2}{30} \frac{(k_b T_\mathrm{CMB0})^4}{\hbar^3 c^5} \cdot \frac{8\pi G}{3H_0^2}, \label{eq:M1:theory:omega_gamma0_T} \\
    \onun &= N_\mathrm{eff} \cdot \frac{7}{8} \cdot \closed{\frac{4}{11}}^{4/3} \ogn. \label{eq:M1:theory:omega_nu0_T}
\end{align}
%
%
The value of $\oln$ is fixed by the requirement that $H(a=1)=H_0$, yielding 
\begin{equation}
    \oln = 1 - (\omn + \oradn + \okn). \label{eq:M1:theory:omega_lambda_0_normalization_condition}
\end{equation}
%

We also introduce the scaled Hubble factor, $\H\equiv aH$. Rather than working with the scale factor, $a(t)$, we will mainly be working with the logarithm of the scale factor 
\begin{equation} \label{eq:M1:theory:x_dx_definitions}
    x\equiv \ln a,\quad '\equiv \dv{x}. 
\end{equation}
%
The resulting expression for $\Hx$ is thus  
\begin{equation} \label{eq:M1:theory:Hp_of_x}
    \Hx = H_0 \sqrt{\omn e^{-x} + \oradn e^{-2x} + \okn + \oln e^{2x}}. 
\end{equation}
%
This form of the Hubble factor is the one we will focus on for the majority of this report. In terms of $\Hx$, the value of the density parameters can be obtained at any given $x$, with 
\begin{align}
    \ok(x) &= \frac{\okn}{\Hx^2 / H_0^2}, \label{eq:M1:theory:omega_k_of_x}\\ 
    \om(x) &= \frac{\omn}{e^x \Hx^2 / H_0^2}, \label{eq:M1:theory:omega_m_of_x}\\ 
    \orad(x) &= \frac{\oradn}{e^{2x} \Hx^2 / H_0^2}, \label{eq:M1:theory:omega_r_of_x}\\ 
    \ol(x) &= \frac{\oln}{e^{-2x} \Hx^2 / H_0^2}, \label{eq:M1:theory:omega_L_of_x}    
\end{align}

From these expressions, we can identify the epochs during which the Universe was dominated by an equal amount of matter and radiation, and by an equal amount of matter and dark energy. These epochs are defined by the time when $\om=\orad$ and $\om=\ol$, respectively, and are a valuable asset towards understanding the physics governing the evolution of the Universe. Another time of interest is the onset of acceleration, defined as the time when $\ddot{a}=0$. In terms of $\H$ and $x$, this corresponds to 
\begin{equation}
    \ddot{a} = \dv{x}{t}\dv{\dot{a}}{x}=\dv{\ln a}{t}\dv{\Hx}{x} = e^{-x} \Hx \dv{\Hx}{x}. \label{eq:M1:theory:acceleration_onset}
\end{equation}   
In Sect. \ref{subsec:M1:theory:analytical_solutions} we will derive an expression for $\H'(x)$. 

%======= eta and t ===============
\subsubsection{Conformal time} \label{sssec:M1:theory:conformal_time}
We now want to relate the Hubble factor to some time variables. The main one we will consider is the conformal time, $\eta$. It is a measure of the distance light has been able to travel since $t=0$, where $t$ is the cosmic time. Using its definition in terms of $t$ \cite[Eq. (2.90)]{Dodelson}, we can express it in terms of $x$ as 
\begin{equation}
    \eta = \int_0^t \frac{c \, \dd t'}{a(t')} = \int_{-\infty}^{x'} \frac{c\,\dd x'}{\H(x')}. \label{eq:M1:theory:eta_of_x_integral_expression}
\end{equation}
%
This leads us to the following differential equation that we will solve numerically 
\begin{equation}
    \dv{\eta}{x} = \frac{c}{\Hx}. \label{eq:M1:theory:eta_ODE}
\end{equation}
%
The initial condition we have is $\eta(-\infty)=0$. Noting from Eq. \eqref{eq:M1:theory:Hp_of_x} that $\Hx\to H_0\sqrt{\oradn}e^{-x}$ as $x\to-\infty$, we get an analytical approximation for the initial condition of $\eta$ at early times 
\begin{align}
    \eta(x_\mathrm{start}) \approx \int_{-\infty}^{x_\mathrm{start}} \frac{c\,\dd x'}{H_0\sqrt{\oradn}} e^{x'} = \frac{c}{\H(x_\mathrm{start})}. \label{eq:M1:theory:eta_of_xstart_analytical_approximation}
\end{align} 
%
Note that $\eta(x)\Hx/c\to 1$ at low $x$, which provides a natural way of validating our implementation.    

For the cosmic time, $t$, starting from $H=\dot{a}/a$ and applying the chain rule yields the desired differential equation for $t(x)$, which we will solve numerically, 
\begin{equation}
    \dv{t}{x} = \frac{1}{H(x)}. \label{eq:M1:theory:cosmic_time_ODE}
\end{equation}
To get an initial condition for $t$, we consider the radiation dominating era, with the following integral expression 
\begin{equation}
    t(x) = \int_{-\infty}^x \frac{\dd x'}{H(x')}. \label{eq:M1:theory:t_of_x_integral_expression}
\end{equation} 
%
Comparing with Eq. \eqref{eq:M1:theory:eta_of_xstart_analytical_approximation}, we see that the two integrands only differ by a factor $e^x$. The initial condition for $t$ is therefore easily seen to be  
\begin{equation}
    t(x_\mathrm{start}) = \frac{1}{2H(x_\mathrm{start})}. \label{eq:M1:theory:t_of_xstart_analytical_approximation}
\end{equation} 

%============== Supernova fitting =================
\subsubsection{Distance measures} 
The supernova data we will study has distances measured in terms of luminosity distance, $d_L$. Expressing it in terms of the angular distance, $d_A=ar$, it becomes 
\begin{equation}
    d_L(a) = \frac{d_A}{a^2} = \frac{r}{a} \implies d_L(x)=e^{-x}r. \label{eq:m1:theory:dL_of_a_general expression}
\end{equation} 
Here, $r$ represents the radial coordinate of the emitted photon. To get an expression for $r$, we consider a photon's line-element in spherical coordinates,
\begin{equation}
    ds^2 = -c^2 dt^2 + a^2 \closed{\frac{dr^2}{1-kr^2} + r^2 d\theta^2 + r^2 \sin^2\theta d\phi^2}. \label{eq:M1:theory:FLRW_general_nonflat_spherical}
\end{equation} 
For photons travelling radially towards us, we have $\dd \theta=\dd\phi=0$. Since $ds^2=0$ for photons, integrating the line-element of a photon emitted at, $(t,r)$, reaching an observer at $(t_0, 0)$, yields 
\begin{equation}
    \int_0^r \frac{\dd r'}{\sqrt{1-k r'}} = \int_t^{t_0} \frac{c\,\dd t}{a}. \label{eq:M1:theory:comoving_distance_integral_expression}
\end{equation} 
%
The RHS of Eq. \eqref{eq:M1:theory:comoving_distance_integral_expression} is known as the co-moving distance, $\chi$, which in terms of conformal time is given as  
\begin{equation}
    \chi = \int_t^{t_0} \frac{c\,\dd t}{a} = \int_x^0 \frac{c\,\dd x'}{\H(x')} = \eta(0) - \eta(x). \label{eq:M1:theory:chi_eta0_minus_eta_of_x}
\end{equation}
%
Solving Eq. \eqref{eq:M1:theory:comoving_distance_integral_expression} with respect to $r$, we get
\begin{equation}
    r = \begin{cases}
        \chi\cdot \frac{\sin\closed{\sqrt{\abs{\okn}} H_0 \chi / c}}{\closed{\sqrt{\abs{\okn}} H_0 \chi / c}},\quad &\okn < 0, \\ 
        \chi,\quad &\okn=0, \\ 
        \chi\cdot \frac{\sinh\closed{\sqrt{\abs{\okn}} H_0 \chi / c}}{\closed{\sqrt{\abs{\okn}} H_0 \chi / c}},\quad &\okn > 0.
    \end{cases}
\end{equation} 
Eq. \eqref{eq:m1:theory:dL_of_a_general expression} can now be used to compute $d_L$, and the expression to use depends on the curvature.     


%======= Sanity checks ===========
\subsubsection{Analytical solutions} \label{subsec:M1:theory:analytical_solutions}
In \secref{sssec:M1:theory:conformal_time} we discussed how $\eta(x)$ can be used to test our implementation in the radiation dominating era. To test our solutions in other regimes, we will need the first and second derivative of $\Hx$. To simplify the resulting expressions, we define the function, $g(x)$, as the derivative of the term inside the square root in Eq. \eqref{eq:M1:theory:Hp_of_x}, namely 
\begin{equation}
    g(x) \equiv -\omn e^{-x} -2\oradn e^{-2x} + 2\oln e^{2x}. \label{eq:M1:theory:g_of_x} 
\end{equation} 
%
The first two derivatives of $\Hx$ are easily seen to be   
\begin{align} 
    \dv{\H(x)}{x} &= \frac{H_0^2}{2\Hx}g(x), \label{eq:M1:theory:dHp_dx} \\
    \dv[2]{\Hx}{x} &= \frac{H_0^2}{2\Hx}\bracket{g'(x) - \frac{1}{2}\closed{\frac{H_0 g(x)}{\Hx}}^2}. \label{eq:M1:theory:ddHp_ddx}
\end{align}
%

Now we will consider the situation where the Universe is dominated by a single fluid with a constant EoS parameter, $w$. In that case we have $H(t)^2 \propto \rho_i(t)^2$ \cite[Eq. (3.13)]{Dodelson}. Using Eq. \eqref{eq:M1:theory:rho_i_eos_dependence}, the Hubble parameter expressed in terms of $w_i$ becomes  
\begin{equation}
    H(t)^2 \propto a^{-3(1+w)}\implies \Hx = c_1 e^{-\frac{3}{2}(1+w)x}, \label{eq:M1:theory:Hx_eos_wi}
\end{equation}
where $c_1$ is some constant. The reason for doing this, is that both $c_1$ and the exponential factor drops out when we consider $\H'(x)/\Hx$ and $\H''(x)/\Hx$. For different values of $w_i$, $\H'(x)/\Hx$ becomes    
\begin{equation} \label{eq:M1:theory:dH_dx_over_H_w}
    \frac{1}{\Hx}\dv{\H}{x} = - \frac{1+3w}{2} = 
    \begin{cases}
        -1,\quad & w=1/3, \\ 
        -1/2, \quad & w=0, \\ 
        1, \quad & w=-1.
    \end{cases} 
\end{equation}  
%
Similarly, the expression for $\H''(x)/\Hx$ becomes  
\begin{equation} \label{eq:M1:theory:ddH_ddx_over_H_w}
    \frac{1}{\Hx^2}\dv[2]{\H}{x} = \frac{(1+3w)^2}{2} = 
    \begin{cases}
        1,\quad & w=1/3 \\ 
        1/4,\quad & w=0 \\ 
        1,\quad & w=-1 
    \end{cases}
\end{equation}
%  
Equations \eqref{eq:M1:theory:dH_dx_over_H_w} and \eqref{eq:M1:theory:ddH_ddx_over_H_w} offer a means to evaluate the accuracy of our numerical solution at different regimes. Each density parameter evolve differently with $x$, as seen from Eqs. \eqref{eq:M1:theory:omega_k_of_x}-\eqref{eq:M1:theory:omega_L_of_x}. Certain ranges of $x$-values will therefore closely resemble a Universe that is dominated by a single fluid. By computing $\H$, $\H'$, and $\H''$, we can examine whether these quantities exhibit the expected behaviour. This allows us to assess the validity of our model and ensure that it is consistent with the underlying physical principles.


\subsection{Implementation details}\label{ssec:M2:implementations} 
To compute $\Xe$, we use the Saha equation initially, as it is a good approximation at early times when $\Xe \approx 1$. This is also the regime where the Peebles equation is unstable, and we therefore consider the Saha equation for $\Xe>\Xe^\mathrm{tol}$. Once we reach $\Xe<\Xe^\mathrm{tol}$ we use the final value from the Saha equation as our initial condition to solve the Peebles equation. The Peebles equation is then used all the way to today, at $x=0$. For this report, we choose $\Xe^\mathrm{tol}=0.99$. For comparison with the Saha equation alone, we set $\Xe^\mathrm{tol}=10^{-6}$. Once $\Xe$ is computed we get $\nex$ from \Eqref{eq:M2:theory:X_e_definition}. 

\subsubsection{Solving the Saha Equation}\label{sssec:M2:implementations:solving_saha}
Solving the Saha equation is done by solving a quadratic formula for $\Xe$. At early times, however, the RHS of \Eqref{eq:M2:theory:saha_equation} will be enormous, and may cause numerical errors when solving the quadratic formula. To avoid this, we use the first order approximation $\sqrt{1+x}\approx 1 + \frac{x}{2}$ for $\abs{x}\ll1$ at early times. The Saha equation is thus implemented as   
\begin{equation}
    \Xe = \begin{cases}
        1, \quad & y>10^7, \\
        \frac{y}{2}\bclosed{-1 + \sqrt{1 + 4/y}},\quad & y\leq 10^7,
    \end{cases}
\end{equation}  
where $y$ refers to the RHS of \Eqref{eq:M2:theory:saha_equation}. Since $\Xe$ is strictly positive, we have omitted the negative solution. The exact value of $10^7$ is chosen to ensure both $\Xe \ngtr 1$, and $\Xe\nless\Xe^\mathrm{tol}$ when the quadratic formula is to be used.

\subsubsection{Solving the Peebles Equation}\label{sssec:M2:implementations:solving_peebles}
To solve \Eqref{eq:M2:theory:Xe_peebles_ODE} numerically, we follow the same procedure as we did for $\eta(x)$, but for the initial condition we use the final value of $\Xe$ that we obtained from the Saha equation. 

At late times, when the baryon temperature gets low, the exponent term in \Eqref{eq:M2:theory:peebles:beta2} for $\beta^{(2)}(T_b)$ become sufficiently large to yield an overflow. However, this is also where $\beta(\Tb)\to0$, due to its exponential factor (\Eqref{eq:M2:theory:peebles:beta}). This exponential factor causes $\beta^{(2)}(T_b)\to0$ at late times. To avoid overflow, we implement the equation for $\beta^{(2)}(T_b)$ as 
\begin{equation}
    \beta^{(2)}(T_b) = \begin{cases}
        0,\quad &\epsn/\Tb > 200, \\
        \beta(\Tb) e^{3\epsn/4\Tb},\quad &\epsn/\Tb \leq 200.
    \end{cases}
\end{equation} 
 

\subsubsection{Optical depth and visibility function} \label{sssec:M2:implementations:optical_depth}
With $\ne$ we can then solve \Eqref{eq:M2:theory:tau_ODE} for $\tau(x)$ with the aforementioned initial condition of $\tau(x=0)=0$. We therefore integrate backwards, starting from $x=0$. The visibility function is now easily obtained, as $\tau'(x)$ is given analytically by \Eqref{eq:M2:theory:tau_ODE}. From this, we get immediately $\gx$. 

We also need the second derivative of $\tau$, as well as the first two derivatives of $\g$. For $\tau''(x)$, we compute it from numerically differentiating the $\tau'(x)$ data. \footnote{Couldn't we just compute it analytically? I get a tiny "bump" in $\tau''$ near $X_e^\mathrm{tol}$, and I don't know if this will cause problems later on, and therefore didn't want to spend too much time on it if wasn't needed.} We use this to compute 
\begin{equation} \label{eq:M2:implementations:dg_dx}
    \g'(x) = \vclosed{\tau'(x)^2 - \tau''(x)} e^{-\tau(x)}, 
\end{equation}
and obtain $\g''(x)$ by numerically differentiating $\g'(x)$. This is done to avoid potential errors that may occur when numerically computing the second derivative of $\g(x)$, if $\g(x)$ is somewhat ill-behaved at certain times. \footnote{I don't know if this is useful, or helps, at all. I don't know if we care about the derivatives at $x$-values where this might be an issue.} 

\subsubsection{Determining the time of recombination and decoupling} \label{sssec:M2:implementations:determining_the_time_of_recombination_and_decoupling}
To estimate the times when recombination takes place, we define this as 
\begin{equation}
    \Xe(x=x_\mathrm{recombination}) = 0.1,  
\end{equation}
where the value $\Xe=0.1$ is chosen arbitrarily. 

For decoupling, we find $x_\mathrm{decoupling}$ as the point where $\g'(x_\mathrm{decoupling})=0$. However, we limit the time values to $x\in[x_0\pm0.1]$, where $x_0$ is defined by $\tau(x_0)=1$. This is done to avoid $x$ values where $\g(x)=0$. 

\subsection{Results}\label{ssec:M4:results}


\subsubsection{CMB power spectrum} \label{sssec:M4:results:angular_power_spectrum}
The angular power spectrum is shown in \figref{fig:M4:results:peaks_and_troughs_cells}. We also include data from \cite{Planck2020} for low $\ell$ values. At higher $\ell$, neutrinos, helium and reionization becomes important, and the predicted value of $\cl$ would deviate from that of observations \pnote{Write about how/why these things affect it.} To understand the physics behind the CMB power spectrum, we have also drawn vertical lines showing the peaks (red) and the troughs (blue) in the spectrum. 

The peaks we observe in the power spectrum are a result of the oscillations of the photon baryon plasma. When photons decouple from baryons, modes at different scales will be in different states. If the plasma is fully compressed upon recombination, the photons have to escape from deep potential wells, and will therefore be redshifted. Similarly, if a mode is fully decompressed when recombination happens, the photons can escape without much energy loss. Since $\cl$ depends on the absolute value of the photons multipoles, both of these cases will result in a peak of the power spectrum. The troughs therefore correspond to modes for which the temperature perturbation is approximately zero.  

As seen in \secref{ssec:M3:results}, the size of the modes strongly affect their behaviour before decoupling. Small scale modes will undergo several oscillations before recombination occurs. The number of oscillations a mode undergoes before decoupling decreases for larger scale modes, and for modes larger than the horizon during recombination, we expect no oscillations to take place. Thus, the peak at $\ell=205$ corresponds to a mode that just began to oscillate, with recombination taking place when it was at its maximum. As $\ell$ is decreased from $205$, the modes are gradually less compressed, until we reach scales of modes that had not entered the horizon before decoupling. 


\begin{figure}[ht!]
    \includefig{peaks_and_troughs_cells\pspecresolution}
    \caption{The predicted CMB power spectrum (black) compared with data from Planck at large scales. \note{Fix y-label. Should indicate $\tcmb$}}
    \label{fig:M4:results:peaks_and_troughs_cells}
\end{figure}

To see that the peaks and troughs in \figref{fig:M4:results:peaks_and_troughs_cells} actually correspond to modes where $\Thn$ is at extrema, we plot $\Thn(k,x)$ as a function of $x$ for $k=\lp/\etan$ and $k=\lt/\etan$, shown in \figref{fig:M4:results:theta0_at_peaks_and_troughs}. We also mark the region of $x\sim x\rec$. \note{Comment on why recombination isn't instantaneous.} As evident from the figure, the first three peaks of $\cl$ correspond to modes on scales that are at maxima. The troughs occur at scales that are essentially zero deviation from the initial temperature perturbation. Modes with wavenumber $k\sim\lt/\etan$ give roughly zero contribution to the power spectrum, but there are many modes of different wavenumbers with a non-zero contribution, resulting in the troughs we see in the power spectrum, and not $C_{\lt}=0$. Another important feature, is that the maxima of $\Thn(x)$ in the upper panel of \figref{fig:M4:results:theta0_at_peaks_and_troughs} don't occur at $x=x\rec$. In general, inhomogeneities on scale $k$ contributes to $\cl$ at slightly lower value of $\ell$ than $\ell=k\eta_0$. One reason for this, is that $\jell(x)$ peaks when $\ell$ is slightly smaller than $x$. 
\begin{figure}[ht!] 
    \includefig{Theta0_at_peaks_and_troughs}
    \caption{Computed monopoles for of scale $k=\ell/\eta_0$ for different values of $\ell$. The upper panel are $\ell$ values where the power spectrum peaks, and the lower panel are the $\ell$ values of the power spectrum troughs.}
    \label{fig:M4:results:theta0_at_peaks_and_troughs}
\end{figure}
The next prominent feature in \figref{fig:M4:results:peaks_and_troughs_cells} are the damping of the peaks at small scales. During tight coupling, photons and baryons do not behave exactly as one fluid. In reality, photons travel a certain distance between scattering events, essentially performing random walks. During a Hubble time $H^{-1}$, a photon will scatter $N=n_e \sigma_T / H$ times, and with a mean free path of $\mfp$, the mean distance a photon travels during a Hubble time is 
\begin{equation}
    \lambda_D = \mfp \sqrt{N} = \sqrt{n_e \sigma_T H}^{-1}.
\end{equation}

As previously discussed, recombination is not an instantaneous event, 

\begin{figure}[ht!]
    \includefig{cells_components\pspecresolution}
    \caption{The total angular power spectrum (dashed blue curve), compared to power spectra computed from individual terms in \Eqref{eq:M4:theory:source_function_LOS}. For each term in the figure, the remaining terms in \Eqref{eq:M4:theory:source_function_LOS} have been set to zero.}
    \label{fig:M4:results:cells_components}
\end{figure}

\subsubsection{Matter power spectrum} \label{sssec:M4:results:matter_power_spectrum}
\begin{figure}[ht!]
    \includefig{matterPS_nk1000}
    \caption{Computed matter power spectrum (solid blue curve), compared with observational data from \note{cite}. The equality scale, $\keq$, is marked as the vertical black line.}
    \label{fig:M4:results:matterPS_nk1000}
\end{figure}




\subsubsection{Photon multipoles} \label{sssec:M4:results:photons_multipoles}
\begin{figure}[ht!]
    \includefig{integrand_thetas\pspecresolution\ellintegrand}
    \caption{The term involved in the integrand of \Eqref{eq:M4:theory:CMB_power_spectrum}, for different values of $\ell$. Note that we scale each integrand by $\ell(\ell+1)$, just as we did for $\cl$. This is done to see each of the terms simultaneously.}
    \label{fig:M4:results:integrand_thetas}
\end{figure}


\begin{figure}[ht!]
    \includefig{thetas\pspecresolution\ellthetas}
    \caption{The transfer function from \Eqref{eq:M4:theory:Theta_ell_LOS_integration} plotted for different values of $\ell$.}
    \label{fig:M4:results:thetas}
\end{figure}