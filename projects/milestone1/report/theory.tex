

\subsection{Theory}\label{ssec:M1:theory}

\subsubsection{Density parameters}

The Friedmann equation can be written in terms of density parameters, $\Omega_i\equiv \rho_i/\rho_c$, where $\rho_c\equiv 3H^2/8\pi G$ is the critical density. We will assume that a given species, $i$, of the Universe can be described by a constant equation of state, $w_i \equiv P_i/\rho_i$, where $P_i$ denotes the pressure. The density of a species evolves as \cite[Eq. (2.72)]{Dodelson}
\begin{equation}
    \rho_i(t)\propto a(t)^{-3(1+w_i)}. \label{eq:M1:theory:rho_i_eos_dependence}
\end{equation}
%
%
For baryons and cold dark matter (CDM), we have $w=0$, for photons and massless neutrinos \note{(Write massless assumption)} we have $w=1/3$ and for a cosmological constant we have $w=-1$. Including a curvature parameter $\Omega_k$, with $w=-1/3$, the Friedmann Equation can be written as \cite[Eq. (3.14)]{Dodelson}  
\begin{equation}
    H = H_0 \sqrt{\omn a^{-3} + \oradn a^{-4} + \okn a^{-2} + \oln}, \label{eq:M1:theory:Friedmann_H_omegas}
\end{equation}
where $H\equiv\dot{a}/a$ is the Hubble parameter, with the dot denoting a derivative with respect to cosmic time, $t$. For brevity, we have expressed the density parameters of matter and radiation as $\omn=\obn+\ocdmn$ and $\oradn=\ogn+\onun$, respectively. A subscript $0$ indicates the value of a given parameter today, at $a=1$. The density parameters of radiation follow from the CMB temperature, and are given by 
\begin{align}
    \ogn &= 2 \cdot \frac{\pi^2}{30} \frac{(k_b T_\mathrm{CMB0})^4}{\hbar^3 c^5} \cdot \frac{8\pi G}{3H_0^2}, \label{eq:M1:theory:omega_gamma0_T} \\
    \onun &= N_\mathrm{eff} \cdot \frac{7}{8} \cdot \closed{\frac{4}{11}}^{4/3} \ogn. \label{eq:M1:theory:omega_nu0_T}
\end{align}
%
%
The value of $\oln$ is fixed by the requirement that $H(a=1)=H_0$, yielding 
\begin{equation}
    \oln = 1 - (\omn + \oradn + \okn). \label{eq:M1:theory:omega_lambda_0_normalization_condition}
\end{equation}
%

We also introduce the scaled Hubble factor, $\H\equiv aH$. Rather than working with the scale factor, $a(t)$, we will mainly be working with the logarithm of the scale factor 
\begin{equation} \label{eq:M1:theory:x_dx_definitions}
    x\equiv \ln a,\quad '\equiv \dv{x}. 
\end{equation}
%
The resulting expression for $\Hx$ is thus  
\begin{equation} \label{eq:M1:theory:Hp_of_x}
    \Hx = H_0 \sqrt{\omn e^{-x} + \oradn e^{-2x} + \okn + \oln e^{2x}}. 
\end{equation}
%
Once $\Hx$ is known, we can compute the value of the density parameters at any given $x$, with 
\begin{align}
    \ok(x) &= \frac{\okn}{\Hx^2 / H_0^2}, \label{eq:M1:theory:omega_k_of_x}\\ 
    \om(x) &= \frac{\omn}{e^x \Hx^2 / H_0^2}, \label{eq:M1:theory:omega_m_of_x}\\ 
    \orad(x) &= \frac{\oradn}{e^{2x} \Hx^2 / H_0^2}, \label{eq:M1:theory:omega_r_of_x}\\ 
    \ol(x) &= \frac{\oln}{e^{-2x} \Hx^2 / H_0^2}, \label{eq:M1:theory:omega_L_of_x}    
\end{align}
With these expressions, we can determine when the Universe was dominated by an equal amount of matter and radiation, and by an equal amount of matter and cosmological constant. These can be found numerically, as $\orad(x)=\om(x)$ and $\om(x)=\ol(x)$, respectively. Another time of interest is the onset of acceleration, defined as the time when $\ddot{a}=0$. In terms of $\H$ and $x$, this corresponds to 
\begin{equation}
    \ddot{a} = \dv{x}{t}\dv{\dot{a}}{x}=\dv{\ln a}{t}\dv{\Hx}{x} = e^{-x} \Hx \dv{\Hx}{x}. \label{eq:M1:theory:acceleration_onset}
\end{equation}   
In Sect. \ref{subsec:M1:theory:analytical_solutions} we will derive an expression for the derivative of $\H$ that we can use. 

%======= eta and t ===============
\subsubsection{Conformal time}
Having considered the main quantities governing the evolution of the background cosmology in terms of $x$, we want to relate these quantities to some time variables. One of the main time variables we will be working with is the conformal time, $\eta$, which is a measure of the distance light have been able to travel since $t=0$, where $t$ is the cosmic time. Using its definition in terms of $t$ \cite[Eq. (2.90)]{Dodelson}, we can express it in terms of $x$ as 
\begin{equation}
    \eta = \int_0^t \frac{c \, \dd t'}{a(t')} = \int_{-\infty}^{x'} \frac{c\,\dd x'}{\H(x')}. \label{eq:M1:theory:eta_of_x_integral_expression}
\end{equation}
%
This leads us to the following differential equation that we will solve numerically 
\begin{equation}
    \dv{\eta}{x} = \frac{c}{\Hx}. \label{eq:M1:theory:eta_ODE}
\end{equation}
%
The initial condition we have is $\eta(-\infty)=0$. Noting from Eq. \eqref{eq:M1:theory:Hp_of_x} that $\Hx\to H_0\sqrt{\oradn}e^{-x}$ as $x\to-\infty$, we get an analytical approximation for the initial condition of $\eta$ at early times 
\begin{align}
    \eta(x_\mathrm{start}) \approx \int_{-\infty}^{x_\mathrm{start}} \frac{c\,\dd x'}{H_0\sqrt{\oradn}} e^{x'} = \frac{c}{\H(x_\mathrm{start})}. \label{eq:M1:theory:eta_of_xstart_analytical_approximation}
\end{align} 
%
\note{Move sentence below to check section.} 
From this, we also get a way of checking that our solution is reasonable, by checking if $\eta(c)\Hx/c=1$ at low $x$.    
% 

In addition to $\eta$, we also want to know the value of $t$ at different points in the Universe's evolution. By definition, $H=\dot{a}/a$, and from the chain rule we obtain a differential equation for $t(x)$ which we can solve numerically, 
\begin{equation}
    \dv{t}{x} = \frac{1}{H(x)}. \label{eq:M1:theory:cosmic_time_ODE}
\end{equation}
To get an initial condition for $t$, we consider the radiation dominating era, with the following integral expression 
\begin{equation}
    t(x) = \int_{-\infty}^x \frac{\dd x'}{H(x')}. \label{eq:M1:theory:t_of_x_integral_expression}
\end{equation} 
%
Comparing with Eq. \eqref{eq:M1:theory:eta_of_xstart_analytical_approximation}, we see that the two integrands only differ by a factor $e^x$. The initial condition for $t$ is therefore easily seen to be  
\begin{equation}
    t(x_\mathrm{start}) = \frac{1}{2H(x_\mathrm{start})}. \label{eq:M1:theory:t_of_xstart_analytical_approximation}
\end{equation} 

%============== Supernova fitting =================
\subsubsection{Distance measures} 
\note{Rewrite this, express with x.}

For the supernova fitting, we will need a measure of the luminosity distance, $d_L$, defined as 
%
\begin{equation}
    d_L(a) = \frac{d_A}{a^2} = \frac{r}{a}, \label{eq:m1:theory:dL_of_a_general expression}
\end{equation} 
where $d_A=ar$ is the angular distance of the source at a distance $r$ away from us. Photons move on $0$-geodesics, and from the line-element in spherical coordinates,
\begin{equation}
    ds^2 = -c^2 dt^2 + a^2 \closed{\frac{dr^2}{1-kr^2} + r^2 d\theta^2 + r^2 \sin^2\theta d\phi^2}, \label{eq:M1:theory:FLRW_general_nonflat_spherical}
\end{equation} 
we get the co-moving distance by integrating the line element of radially moving photons. For a photon emitted at, $(t,r)$, reaching an observer at $(t_0, 0)$, we get 
\begin{equation}
    \int_0^r \frac{\dd r'}{\sqrt{1-k r'}} = \int_t^{t_0} \frac{c\,\dd t}{a}. \label{eq:M1:theory:comoving_distance_integral_expression}
\end{equation} 
\note{(Define $\okn=-kc^2/H_0^2$)}
%

The RHS is the co-moving distance, $\chi$, which in terms of conformal time is given as  
\begin{equation}
    \chi = \int_t^{t_0} \frac{c\,\dd t}{a} = \int_x^0 \frac{c\,\dd x'}{\H(x')} = \eta(0) - \eta(x). \label{eq:M1:theory:chi_eta0_minus_eta_of_x}
\end{equation}
%
Solving Eq. \eqref{eq:M1:theory:comoving_distance_integral_expression} with respect to $r$, we get (\note{check punctuation after cases})
\begin{equation}
    r = \begin{cases}
        \chi\cdot \frac{\sin\closed{\sqrt{\abs{\okn}} H_0 \chi / c}}{\closed{\sqrt{\abs{\okn}} H_0 \chi / c}},\quad &\okn < 0 \\ 
        \chi,\quad &\okn=0 \\ 
        \chi\cdot \frac{\sinh\closed{\sqrt{\abs{\okn}} H_0 \chi / c}}{\closed{\sqrt{\abs{\okn}} H_0 \chi / c}},\quad &\okn > 0
    \end{cases}
\end{equation} 


%======= Sanity checks ===========
\subsubsection{Analytical solutions (Working title)} \label{subsec:M1:theory:analytical_solutions}
In order to test our solutions we will need the first and second derivative of $\Hx$. To simplify the resulting expressions, we define the function, $g(x)$, as the derivative of the term inside the square root in Eq. \eqref{eq:M1:theory:Hp_of_x}, namely 
\begin{equation}
    g(x) \equiv -\omn e^{-x} -2\oradn e^{-2x} + 2\oln e^{2x}. \label{eq:M1:theory:g_of_x} 
\end{equation} 
%
The first two derivatives of $\Hx$ are   
\begin{align} 
    \dv{\H(x)}{x} &= \frac{H_0^2}{2\Hx}g(x), \label{eq:M1:theory:dHp_dx} \\
    \dv[2]{\Hx}{x} &= \frac{H_0^2}{2\Hx}\bracket{g'(x) - \frac{1}{2}\closed{\frac{H_0 g(x)}{\Hx}}^2}. \label{eq:M1:theory:ddHp_ddx}
\end{align}
%

Now we will consider the situation where the Universe is dominated by a single fluid with a constant equation of state parameter, $w$. The density of that fluid evolves according to \eqref{eq:M1:theory:rho_i_eos_dependence}, which we express in terms of the Hubble parameter, 
\begin{equation}
    H(t)^2 \propto a^{-3(1+w)}\implies \Hx = c_1 e^{-\frac{3}{2}(1+w)x}, \label{eq:M1:theory:Hx_eos_wi}
\end{equation}
where $c_1$ is some constant. Now we obtain an analytical expression for $\H'(x)/\Hx$ in terms of $w$ only, as other factors cancel out. Using that $w=1/3,\,0,\,-1$ for radiation, matter and the cosmological constant, we get   
\begin{equation} \label{eq:M1:theory:dH_dx_over_H_w}
    \frac{1}{\Hx}\dv{\H}{x} = - \frac{1+3w}{2} = 
    \begin{cases}
        -1,\quad & w=1/3 \\ 
        -1/2, \quad & w=0 \\ 
        1, \quad & w=-1
    \end{cases} 
\end{equation}  
%
Similarly, the expression for $\H''(x)/\Hx$ becomes  
\begin{equation} \label{eq:M1:theory:ddH_ddx_over_H_w}
    \frac{1}{\Hx^2}\dv[2]{\H}{x} = \frac{(1+3w)^2}{2} = 
    \begin{cases}
        1,\quad & w=1/3 \\ 
        1/4,\quad & w=0 \\ 
        1,\quad & w=-1 
    \end{cases}
\end{equation}
%  
Equations \eqref{eq:M1:theory:dH_dx_over_H_w} and \eqref{eq:M1:theory:ddH_ddx_over_H_w} offer a means to evaluate the accuracy of our numerical solution. By computing the values of $\H$, $\H'$, and $\H''$, we can examine whether these quantities exhibit the expected behaviour in regimes where a single component dominates the Universe. This allows us to assess the validity of our model and ensure that it is consistent with the underlying physical principles.