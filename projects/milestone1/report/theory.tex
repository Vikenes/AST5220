

\subsection{Theory}\label{ssec:M1:theory}

\subsubsection{Density parameters and Hubble factor}

The Friedmann equation can be written in terms of density parameters, $\Omega_i\equiv \rho_i/\rho_c$, where $\rho_c\equiv 3H^2/8\pi G$ is the critical density. The density of a given species, $i$, evolves as \cite[Eq. (2.61)]{Dodelson}
\begin{equation}
    \rho_i(t)\propto a(t)^{-3(1+w_i)}, \label{eq:M1:theory:rho_i_eos_dependence}
\end{equation}
where we have assumed that the equation of state (EoS) parameter, $w_i \equiv P_i/\rho_i$ \cite[Eq. 2.60]{Dodelson}, is constant. $P_i$ denotes the pressure of the species. 
%
We will limit ourselves to consider three types of species in this report: matter, radiation and dark energy. We will only consider baryons and cold dark matter (CDM) for the matter component, which we express as $\omn=\obn+\ocdmn$. The subscript $0$ is used to refer to today's value. The radiation component we consider is $\oradn=\ogn+\onun$, corresponding to photons and neutrinos, respectively. For the dark energy, we only have the cosmological constant $\oln$.   

Matter, radiation and dark energy have densities evolving according to \Eqref{eq:M1:theory:rho_i_eos_dependence} with $w_i=0,\,1/3$ and $-1$, respectively. Neutrinos having $w=1/3$ only holds since we will assume that the neutrinos are massless. Curvature can be described by $w_i=-1/3$, with $\okn\equiv -kc^2/H_0^2$. The parameter $k$ represents the curvature of the Universe, where $k=0$ corresponds to a flat Universe. With these parameters, the Friedmann Equation can be written as \cite[Eq. (3.14)]{Dodelson}  
\begin{equation}
    H = H_0 \sqrt{\omn a^{-3} + \oradn a^{-4} + \okn a^{-2} + \oln}, \label{eq:M1:theory:Friedmann_H_omegas}
\end{equation}
where $H\equiv\dot{a}/a$ is the Hubble parameter, with the dot denoting a derivative with respect to cosmic time, $t$. For the radiation, $\ogn$ and $\onun$ follow from the temperature of the CMB today, $\tcmb$, and the effective number of massless neutrinos, $N_\mathrm{eff}$. They are given by 
\begin{align}
    \ogn &= 2 \cdot \frac{\pi^2}{30} \frac{(k_b T_\mathrm{CMB0})^4}{\hbar^3 c^5} \cdot \frac{8\pi G}{3H_0^2}, \label{eq:M1:theory:omega_gamma0_T} \\
    \onun &= N_\mathrm{eff} \cdot \frac{7}{8} \cdot \pclosed{\frac{4}{11}}^{4/3} \ogn. \label{eq:M1:theory:omega_nu0_T}
\end{align}
%
%
The value of $\oln$ is fixed by the requirement that $H(a=1)=H_0$, yielding 
\begin{equation}
    \oln = 1 - (\omn + \oradn + \okn). \label{eq:M1:theory:omega_lambda_0_normalization_condition}
\end{equation}
%

We also introduce the scaled Hubble factor, $\H\equiv aH$. Rather than working with the scale factor, $a(t)$, we will mainly be working with the logarithm of the scale factor 
\begin{equation} \label{eq:M1:theory:x_dx_definitions}
    x\equiv \ln a,\quad '\equiv \dv{x}. 
\end{equation}
%
The resulting expression for $\Hx$ is thus  
\begin{equation} \label{eq:M1:theory:Hp_of_x}
    \Hx = H_0 \sqrt{\omn e^{-x} + \oradn e^{-2x} + \okn + \oln e^{2x}}. 
\end{equation}
%
This form of the Hubble factor is the one we will focus on for the majority of this report. In terms of $\Hx$, the value of the density parameters can be obtained at any given $x$, with 
\begin{align}
    \ok(x) &= \frac{\okn}{\Hx^2 / H_0^2}, \label{eq:M1:theory:omega_k_of_x}\\ 
    \om(x) &= \frac{\omn}{e^x \Hx^2 / H_0^2}, \label{eq:M1:theory:omega_m_of_x}\\ 
    \orad(x) &= \frac{\oradn}{e^{2x} \Hx^2 / H_0^2}, \label{eq:M1:theory:omega_r_of_x}\\ 
    \ol(x) &= \frac{\oln}{e^{-2x} \Hx^2 / H_0^2}, \label{eq:M1:theory:omega_L_of_x}    
\end{align}

From these expressions, we can identify the epochs during which the Universe was dominated by an equal amount of matter and radiation, and by an equal amount of matter and dark energy. These epochs are defined by the time when $\om=\orad$ and $\om=\ol$, respectively, and are a valuable asset towards understanding the physics governing the evolution of the Universe. Another time of interest is the onset of acceleration, defined as the time when $\ddot{a}=0$. In terms of $\H$ and $x$, this corresponds to 
\begin{equation}
    \ddot{a} = \dv{x}{t}\dv{\dot{a}}{x}=\dv{\ln a}{t}\dv{\Hx}{x} = e^{-x} \Hx \dv{\Hx}{x}. \label{eq:M1:theory:acceleration_onset}
\end{equation}   
In Sect. \ref{subsec:M1:theory:analytical_solutions} we will derive an expression for $\H'(x)$. 

%======= eta and t ===============
\subsubsection{Conformal time} \label{sssec:M1:theory:conformal_time}
We now want to relate the Hubble factor to some time variables. The main one we will consider is the conformal time, $\eta$. It is a measure of the distance light has been able to travel since $t=0$, where $t$ is the cosmic time. Using its definition in terms of $t$ \cite[Eq. (2.90)]{Dodelson}, we can express it in terms of $x$ as 
\begin{equation}
    \eta = \int_0^t \frac{c \, \dd t'}{a(t')} = \int_{-\infty}^{x'} \frac{c\,\dd x'}{\H(x')}. \label{eq:M1:theory:eta_of_x_integral_expression}
\end{equation}
%
This leads us to the following differential equation that we will solve numerically 
\begin{equation}
    \dv{\eta}{x} = \frac{c}{\Hx}. \label{eq:M1:theory:eta_ODE}
\end{equation}
%
The initial condition we have is $\eta(-\infty)=0$. Noting from \Eqref{eq:M1:theory:Hp_of_x} that $\Hx\to H_0\sqrt{\oradn}e^{-x}$ as $x\to-\infty$, we get an analytical approximation for the initial condition of $\eta$ at early times 
\begin{align}
    \eta(x_\mathrm{start}) \approx \int_{-\infty}^{x_\mathrm{start}} \frac{c\,\dd x'}{H_0\sqrt{\oradn}} e^{x'} = \frac{c}{\H(x_\mathrm{start})}. \label{eq:M1:theory:eta_of_xstart_analytical_approximation}
\end{align} 
%
Note that $\eta(x)\Hx/c\to 1$ at low $x$, which provides a natural way of validating our implementation.    

For the cosmic time, $t$, starting from $H=\dot{a}/a$ and applying the chain rule yields the desired differential equation for $t(x)$, which we will solve numerically, 
\begin{equation}
    \dv{t}{x} = \frac{1}{H(x)}. \label{eq:M1:theory:cosmic_time_ODE}
\end{equation}
To get an initial condition for $t$, we consider the radiation dominating era, with the following integral expression 
\begin{equation}
    t(x) = \int_{-\infty}^x \frac{\dd x'}{H(x')}. \label{eq:M1:theory:t_of_x_integral_expression}
\end{equation} 
%
Comparing with \Eqref{eq:M1:theory:eta_of_xstart_analytical_approximation}, we see that the two integrands only differ by a factor $e^x$. The initial condition for $t$ is therefore easily seen to be  
\begin{equation}
    t(x_\mathrm{start}) = \frac{1}{2H(x_\mathrm{start})}. \label{eq:M1:theory:t_of_xstart_analytical_approximation}
\end{equation} 

%============== Supernova fitting =================
\subsubsection{Distance measures} 
The supernova data we will study has distances measured in terms of luminosity distance, $d_L$. Expressing it in terms of the angular distance, $d_A=ar$, it becomes 
\begin{equation}
    d_L(a) = \frac{d_A}{a^2} = \frac{r}{a} \implies d_L(x)=e^{-x}r. \label{eq:M1:theory:dL_of_a_general expression}
\end{equation} 
Here, $r$ represents the radial coordinate of the emitted photon. To get an expression for $r$, we consider a photon's line-element in spherical coordinates,
\begin{equation}
    ds^2 = -c^2 dt^2 + a^2 \pclosed{\frac{dr^2}{1-kr^2} + r^2 d\theta^2 + r^2 \sin^2\theta d\phi^2}. \label{eq:M1:theory:FLRW_general_nonflat_spherical}
\end{equation} 
For photons travelling radially towards us, we have $\dd \theta=\dd\phi=0$. Since $ds^2=0$ for photons, integrating the line-element of a photon emitted at, $(t,r)$, reaching an observer at $(t_0, 0)$, yields 
\begin{equation}
    \int_0^r \frac{\dd r'}{\sqrt{1-k r'}} = \int_t^{t_0} \frac{c\,\dd t}{a}. \label{eq:M1:theory:comoving_distance_integral_expression}
\end{equation} 
%
The RHS of \Eqref{eq:M1:theory:comoving_distance_integral_expression} is known as the co-moving distance, $\chi$, which in terms of conformal time is given as  
\begin{equation}
    \chi = \int_t^{t_0} \frac{c\,\dd t}{a} = \int_x^0 \frac{c\,\dd x'}{\H(x')} = \eta(0) - \eta(x). \label{eq:M1:theory:chi_eta0_minus_eta_of_x}
\end{equation}
%
Solving \Eqref{eq:M1:theory:comoving_distance_integral_expression} with respect to $r$, we get
\begin{equation}
    r = \begin{cases}
        \chi\cdot \frac{\sin\pclosed{\sqrt{\abs{\okn}} H_0 \chi / c}}{\pclosed{\sqrt{\abs{\okn}} H_0 \chi / c}},\quad &\okn < 0, \\ 
        \chi,\quad &\okn=0, \\ 
        \chi\cdot \frac{\sinh\pclosed{\sqrt{\abs{\okn}} H_0 \chi / c}}{\pclosed{\sqrt{\abs{\okn}} H_0 \chi / c}},\quad &\okn > 0.
    \end{cases}
\end{equation} 
\Eqref{eq:M1:theory:dL_of_a_general expression} can now be used to compute $d_L$, and the expression to use depends on the curvature.     


%======= Sanity checks ===========
\subsubsection{Analytical solutions} \label{subsec:M1:theory:analytical_solutions}
In \secref{sssec:M1:theory:conformal_time} we discussed how $\eta(x)$ can be used to test our implementation in the radiation dominating era. To test our solutions in other regimes, we will need the first and second derivative of $\Hx$. To simplify the resulting expressions, we define the function, $g(x)$, as the derivative of the term inside the square root in \Eqref{eq:M1:theory:Hp_of_x}, namely 
\begin{equation}
    g(x) \equiv -\omn e^{-x} -2\oradn e^{-2x} + 2\oln e^{2x}. \label{eq:M1:theory:g_of_x} 
\end{equation} 
%
The first two derivatives of $\Hx$ are easily seen to be   
\begin{align} 
    \dv{\H(x)}{x} &= \frac{H_0^2}{2\Hx}g(x), \label{eq:M1:theory:dHp_dx} \\
    \dv[2]{\Hx}{x} &= \frac{H_0^2}{2\Hx}\bclosed{g'(x) - \frac{1}{2}\pclosed{\frac{H_0 g(x)}{\Hx}}^2}. \label{eq:M1:theory:ddHp_ddx}
\end{align}
%

Now we will consider the situation where the Universe is dominated by a single fluid with a constant EoS parameter, $w$. In that case we have $H(t)^2 \propto \rho_i(t)^2$ \cite[Eq. (3.13)]{Dodelson}. Using \Eqref{eq:M1:theory:rho_i_eos_dependence}, the Hubble parameter expressed in terms of $w_i$ becomes  
\begin{equation}
    H(t)^2 \propto a^{-3(1+w)}\implies \Hx = c_1 e^{-\frac{3}{2}(1+w)x}, \label{eq:M1:theory:Hx_eos_wi}
\end{equation}
where $c_1$ is some constant. The reason for doing this, is that both $c_1$ and the exponential factor drops out when we consider $\H'(x)/\Hx$ and $\H''(x)/\Hx$. For different values of $w_i$, $\H'(x)/\Hx$ becomes    
\begin{equation} \label{eq:M1:theory:dH_dx_over_H_w}
    \frac{1}{\Hx}\dv{\H}{x} = - \frac{1+3w}{2} = 
    \begin{cases}
        -1,\quad & w=1/3, \\ 
        -1/2, \quad & w=0, \\ 
        1, \quad & w=-1.
    \end{cases} 
\end{equation}  
%
Similarly, the expression for $\H''(x)/\Hx$ becomes  
\begin{equation} \label{eq:M1:theory:ddH_ddx_over_H_w}
    \frac{1}{\Hx^2}\dv[2]{\H}{x} = \frac{(1+3w)^2}{2} = 
    \begin{cases}
        1,\quad & w=1/3 \\ 
        1/4,\quad & w=0 \\ 
        1,\quad & w=-1 
    \end{cases}
\end{equation}
%  
Equations \eqref{eq:M1:theory:dH_dx_over_H_w} and \eqref{eq:M1:theory:ddH_ddx_over_H_w} offer a means to evaluate the accuracy of our numerical solution at different regimes. Each density parameter evolve differently with $x$, as seen from Eqs. \eqref{eq:M1:theory:omega_k_of_x}-\eqref{eq:M1:theory:omega_L_of_x}. Certain ranges of $x$-values will therefore closely resemble a Universe that is dominated by a single fluid. By computing $\H$, $\H'$, and $\H''$, we can examine whether these quantities exhibit the expected behaviour. This allows us to assess the validity of our model and ensure that it is consistent with the underlying physical principles.