
\subsection{Implementation details}\label{ssec:M1:implementations} 
To solve the differential equations for $\eta$ and $t$ (\Eqref{eq:M1:theory:eta_ODE} and \eqref{eq:M1:theory:cosmic_time_ODE}) we use the \texttt{C++} library \texttt{GSL} \cite{gough2009gnu}, and use their Runge-Kutta4 solver. From the solution we create a spline of the results for the given $x$ domain we have considered. 

We will consider three different ranges of $x$-values. For the initial testing, we will use ${x\in[\ln 10^{-10},\,5]}$. For fitting cosmological parameters to the supernova data, we will use ${x\in[\ln 10^{-2},\,0]}$. When we want to estimate important times during the cosmic evolution, we will consider $x\in[-10,\,1]$, for increased resolution, as the result may vary by a noticeable amount between step sizes. In all cases, we use $N_x=10^5$ number of points.  

The cosmological parameters we consider assume $\okn=0$. In \secref{sssec:M1:implementations:supernova_fitting} we discuss how we will use supernova data to estimate a value for $\okn$. Curvature is therefore implemented in all the relevant methods, but we set $\okn=0$ when we're not dealing with supernova fitting.  

\subsubsection{Supernova fitting and parameter sampling}\label{sssec:M1:implementations:supernova_fitting}
The supernova data we will use contains $N=31$ data points of luminosity distance, $d_L^\mathrm{obs}(z_i)$, with associated measurement errors, $\sigma_i$, at different redshifts, $z_i\in[0.01,\,1.30]$. Using these measurements, we want to constrain the three-dimensional parameter space 
\begin{equation}
    \mathcal{C} = \Big\{ \hest,\: \omest,\: \okest \Big\}, \label{eq:M1:implementations:supernova_parameter_space}
\end{equation}  
where the hat is used to distinguish the estimated parameters from the fiducial ones. We set $\obn=0.05$, although its particular value is irrelevant for this analysis, which is only sensitive to $\omest=\ocdmn+\obn$. Choosing a value for $\obn$ means that $\omest$ enters via $\ocdmn=\omest-\obn$. Additionally, the small scales we probe here means that neutrinos are irrelevant, since we are well within matter domination, as we will see in \secref{sssec:M1:results:analytical_and_numerical_comparisons}. We therefore set $\neff=0$ for this analysis. 

We will assume that the measurements at different redshifts are normal distributed and uncorrelated. The likelihood function is then given by $L\propto e^{-\chi^2/2}$, where 
\begin{equation}
    \chi^2(\C) = \sum_{i=1}^N \frac{[d_L(z_i, \C) - d_L^\mathrm{obs}(z_i)]^2}{\sigma_i^2}, \label{eq:M1:implementations:chi2_supernova_data}
\end{equation}
is the function we want to minimize. To do this, we will sample parameter values randomly by a Markov chain Monte Carlo (MCMC) process. We also restrict the parameter space to sample, with the following limits:     
\begin{equation} \label{eq:M1:implementations:supernova_parameter_ranges}
    \begin{split}
        0.5 < \hest < 1.5, \\
        0 < \omest < 1, \\
        -1 < \okest < 1.
    \end{split}
\end{equation}   
To generate a new sample, we update each parameter by generating a random number $P\sim\mathcal{N}(0,1)$, and multiplying it by a step size. We will use step sizes of $\Delta \hest = 0.007,\,\Delta \omest=0.05,\,\Delta\okest=0.05$. To determine whether a new configuration should be included in the sample we use the Metropolis algorithm, where we always accept a state if it yields a lower value of $\chi^2$ compared to the previous state that was accepted. If the new value of $\chi^2$ is greater than the old one, we accept it if the ratio of the likelihood functions $L(\chi^2_\mathrm{new})/L(\chi^2_\mathrm{old})>p$, where $p\sim\mathcal{U}(0,1)$. We continue drawing samples until we get a total of $\hat{n}=10^4$ samples. For the samples generated, we omit the first $1000$ samples of the chain from our analysis. 

With our generated samples, we can use the best fit, $\chi^2_\mathrm{min}$, to find the $1\sigma$ and $2\sigma$ confidence regions. For the $\chi^2$ distribution with $3$ parameters, these regions are given by $\chi^2 - \chi^2_\mathrm{min}<3.53$ and $\chi^2 - \chi^2_\mathrm{min}<8.02$, respectively. We will plot the $1\sigma$ and $2\sigma$ constraint in the $(\omn,\oln)$ plane. Since $\oradn<10^{-4}$, $\oln$ can be approximated well by $\oln=1-\omn$. After that we will plot the posterior probability distribution function (PDF) for $H_0$. When presenting the best fit parameters, we include both the standard deviation of each parameter in $\C$, and the quantity $\chi_\mathrm{min}^2/N$, where $N=31$ as previously stated. The latter can be used to validate the goodness of the fit, as this quantity will be close to unity if the fit is good. 

To compare our fit with the Planck data, we will plot $d_L^\mathrm{obs}(z_i)$ together with $d_L^\mathrm{fit}(z)$ and $d_L^\mathrm{Planck}(z)$. We obtain the former by solving the background cosmology with $h,\,\omn,\,\okn$ replaced by the configuration $\hest,\,\omest,\,\okest$ that yielded the lowest value of $\chi^2$.     



