\subsection{Theory}\label{ssec:M3:theory}
In this section we present the equations governing the evolution of the perturbed quantities. We will only present the relevant equations here, which are all obtained from \cite{Dodelson} and \cite{callin}, unless otherwise stated. We refer to \citeauthor{Dodelson} for a detailed derivation of the equations. For the notation, we adapt that of \citeauthor{callin}. 

\subsubsection{Perturbation quantities}\label{sssec:M3:theory:perturbations}
For the perturbed metric we consider the Newtonian gauge \pnote{Explain/cite}, and write it as  
\begin{equation} \label{eq:M3:theory:metric_perturbation}
    g_{\mu\nu} = 
    \begin{pmatrix}
        -(1+2\Psi) & 0 \\
        0 & a^2 \delta_{ij}(1+2\Phi)
    \end{pmatrix},
\end{equation}
where we have introduced the scalar perturbations $\Psi$ and $\Phi$, which are functions of both position and time. For $\Psi=\Phi=0$ we obtain the FLRW metric. Photon perturbations are defined in terms of the relative temperature variations, $\Theta$, via 
\begin{equation} \label{eq:M3:theory:temperature_fluctuations}
    T(\vec{k},\mu,\eta) = \zerothorder{T}(\eta)\bclosed{1 + \Theta(\vec{k},\mu,\eta)},
\end{equation}  
where $\vec{k}$ is the Fourier transformed variable corresponding to position $\vec{x}$, and $\mu\equiv \frac{\vec{k}\cdot\vec{p}}{kp}$. The temperature perturbations only depend on the photon momentum in terms of their directions, and we therefore expand $\Theta$ in terms of multipoles as 
\begin{equation} \label{eq:M3:theory:theta_ell_multipolse_expansion}
    \Theta_\ell = \frac{i^\ell}{2}\int_{-1}^1 \pl(\mu)\Theta(\mu)\,\dd\mu \Leftrightarrow \Theta(\mu) = \sum_{\ell=0}^{\infty}\frac{2\ell+1}{i^\ell}\Theta_\ell \pl(\mu),
\end{equation}
where $\pl(\mu)$ are Legendre polynomials. The most important terms are the monopole, $\Theta_0$, the dipole, $\Theta_1$, and the quadrupole, $\Theta_2$. The monopole is related to the density perturbation of photons, $\dg=4\Theta_0$, and the dipole to the photon velocity, $\vg=-3\Theta_1$. These terms represent the background temperature and our relative motion to the CMB, respectively \note{Check last sentence}.  

For the CDM, we denote the density and velocity perturbations as $\dcdm$ and $\vcdm$, respectively, and similarly denote the baryon perturbations as $\db$ and $\vb$.  

\subsubsection{Perturbation equations}
Having introduced the perturbed quantities of interest, the system of ODEs, given in Fourier space, that we will solve are given in \cite[Eq. (22)]{callin}, where we set all terms involving $\Theta_P$ and $\mathcal{N}$ to zero, as we neglect polarization and neutrinos.   
% \begin{subequations} \label{eq:M3:theory:multipole_ODEs}
%     \begin{align}
%         \Theta_0' =& - \koverhp\Theta_1 - \Phi', \label{eq:M3:theory:multipole_ODEs_Theta0}\\
%         \Theta_1' =& \frac{k}{3\H}\Theta_0 - \frac{2k}{3\H}\Theta_2 + \frac{k}{3\H}\Psi + \tau' \bclosed{\Theta_1 + \frac{1}{3}v_b}, \label{eq:M3:theory:multipole_ODEs_Theta1} \\ 
%         \Theta_\ell' =& \frac{\ell k}{(2\ell+1)\H}\Theta_{\ell-1} - \frac{(\ell+1)k}{(2\ell+1)\H}\Theta_{\ell+1} \nonumber \\
%         & + \tau'\bclosed{\Theta_\ell - \frac{1}{10}\Pi \delta_{\ell,2}},\quad 2\leq\ell<\ell_\mathrm{max}, \label{eq:M3:theory:multipole_ODEs_Theta_ell} \\
%         \Theta_\ell' =& \frac{k}{\H}\Theta_{\ell-1} - c \frac{(\ell+1)}{\H\eta(x)}\Theta_{\ell} + \tau' \Theta_\ell,\quad \ell=\ell_\mathrm{max}, \label{eq:M3:theory:multipole_ODEs_Theta_ell_max} 
%     \end{align}
% \end{subequations}

\begin{subequations} \label{eq:M3:theory:perturbation_ODEs}
    \begin{align}
        \Theta_0' =& - \koverhp\Theta_1 - \Phi', \label{eq:M3:theory:perturbation_ODEs_Theta0}\\
        \Theta_1' =& \frac{k}{3\H}\Theta_0 - \frac{2k}{3\H}\Theta_2 + \frac{k}{3\H}\Psi + \tau' \bclosed{\Theta_1 + \frac{1}{3}v_b}, \label{eq:M3:theory:perturbation_ODEs_Theta1} \\ 
        \Theta_\ell' =& \frac{\ell k}{(2\ell+1)\H}\Theta_{\ell-1} - \frac{(\ell+1)k}{(2\ell+1)\H}\Theta_{\ell+1} \nonumber \\
        & + \tau'\bclosed{\Theta_\ell - \frac{1}{10}\Pi \delta_{\ell,2}},\quad 2\leq\ell<\ell_\mathrm{max}, \label{eq:M3:theory:perturbation_ODEs_Theta_ell} \\
        \Theta_\ell' =& \frac{k}{\H}\Theta_{\ell-1} - c \frac{(\ell+1)}{\H\eta(x)}\Theta_{\ell} + \tau' \Theta_\ell,\quad \ell=\ell_\mathrm{max}, \label{eq:M3:theory:perturbation_ODEs_Theta_ell_max} \\
        \dcdm' =& \frac{k}{\H} \vcdm - 3\Phi', \label{eq:M3:theory:perturbation_ODEs_delta_cdm} \\
        \vcdm' =& - \vcdm \frac{k}{\H} \Psi, \label{eq:M3:theory:perturbation_ODEs_v_cdm} \\
        \db' =& \koverhp \vb - 3\Phi', \label{eq:M3:theory:perturbation_ODEs_delta_b} \\
        \vb' =& -\vb - \koverhp \Psi + \tau' R(3\Theta_1 + \vb), \label{eq:M3:theory:perturbation_ODEs_v_b} \\
        \Phi' =& -\Psi - \frac{k^2}{3\H^2}\Phi + \frac{H_0^2}{2\H^2} \begin{aligned}[t]
            [&\ocdmn a^{-1}\dcdm + \obn a^{-1} \db \\
            &+4\ogn a^{-2} \Theta_0], 
            \end{aligned} \label{eq:M3:theory:perturbation_ODEs_Phi} \\
        \Psi &= -\Phi - \frac{12H_0^2}{k^2 a^2} \ogn \Theta_2. \label{eq:M3:theory:perturbation_ODEs_Psi}
    \end{align}
\end{subequations}
We have $\Pi=\Theta_2$ due to $\Theta_P=0$, and have written $\Pi$ for clarity. 

The expression for $\Theta_\ell'$ is recursive, and we thus need to truncate the series at some point. Truncating the series by setting $\Theta_{\ell_\mathrm{max}}=0$ would lead to numerical errors propagating down to $\Theta_2$. \Eqref{eq:M3:theory:perturbation_ODEs_Theta_ell_max} comes from assuming $\Theta_\ell(k,\eta)\sim j_\ell(k\eta)$ at high $\ell$, where $j_\ell(k\eta)$ is the spherical Bessel function, and using the recurrence relation for $j_\ell$ \citep[see][Eq. (46)-(49)]{callin}. 

The equation for $\Psi$ is an algebraic equation, and its value is inserted in the other equations. Noting that $\Phi'$ is the only derivative of a perturbed quantity found on the RHS of Eqs. \eqref{eq:M3:theory:perturbation_ODEs}, we solve for it first, and insert this value into the other equations. 

At early times, photons and electrons are tightly coupled, and $\tau$ is therefore very high. In this regime, \Eqref{eq:M3:theory:perturbation_ODEs} becomes numerically unstable, so we have to resort to approximations, which we discuss in the next section. 

\subsubsection{Modes \pnote{working title}}\label{sssec:M3:theory:modes}
When studying the evolution of the various perturbation quantities in \Eqref{eq:M3:theory:perturbation_ODEs}, we will consider the evolution for three different modes, $k$, and compare the results. This is because the physics governing the evolution of perturbed quantities depend on the scale of the mode we consider. The short wavelength modes, i.e. high $k$-values, will be affected by causal physics early on, as these modes are small. On the other hand, long wavelength modes, which correspond to small $k$-values, are affected by causal physics later in the Universe. At early times we have $k\eta\ll 1$, and all modes we consider are outside the horizon. As the horizon evolves, more and more modes will enter the horizon. Short wavelength modes will enter the horizon during radiation domination, and evolve differently than long wavelength modes that enter the horizon at later times when the Universe is matter dominated.  

As previously discussed, photons and baryons are tightly coupled before recombination. It is therefore interesting to consider modes entering the horizon long before recombination, during radiation domination, and modes entering after recombination, during matter domination. We will also include intermediate scale modes, that enter the horizon around recombination. The three different modes we will consider are $k\in\nobreak\Bclosed{0.003,\,0.03,\,0.3}/\mathrm{Mpc}$. In \secref{ssec:M3:implementations} we discuss how $x=-8.3$ is chosen as an approximate value for when recombination has started. At this time, we have $k\eta(-8.3)\approx0.35,\,3.5$ and $35$ for our three modes, meaning that the small scales modes have entered the horizon long before recombination, compared to the large scale modes which will enter the horizon long after recombination. \note{Use matter radiation equality rather than recombination time in this section?}  


\subsubsection{Initial conditions} \label{sssec:M3:theory:initial_conditions}
For the initial conditions we follow \cite{winther}, with $f_\nu=0$ as we omit neutrinos\footnote{Note that Winther has a sign error in his expression for the initial condition for $\Psi$}. The set of ODEs we solve are linear, so we can choose a particular normalization, which we adjust for when computing the Power spectrum later on. The initial conditions are      
\begin{subequations} \label{eq:M3:theory:pert_ODE_IC}
    \begin{align}
        \Psi &= -\frac{2}{3}, \label{eq:M3:theory:pert_ODE_IC_Psi} \\
        \Phi &= - \Psi, \label{eq:M3:theory:pert_ODE_IC_Phi} \\
        \dcdm &= \db = -\frac{3}{2}\Psi, \label{eq:M3:theory:pert_ODE_IC_delta_cdm} \\
        \vcdm &= \vb = -\frac{k}{2\H} \Psi, \label{eq:M3:theory:pert_ODE_IC_v_cdm} \\
        \Theta_0 &= -\frac{1}{2}\Psi, \label{eq:M3:theory:pert_ODE_IC_Theta0} \\
        \Theta_1 &= \frac{k}{6\H}\Psi. \label{eq:M3:theory:pert_ODE_IC_Theta1} 
    \end{align}
\end{subequations}
Before the Universe becomes transparent, photons and baryons are tightly coupled, and $\tau'$ is very large, as seen from \figref{fig:M2:results:tau_plot}. Since we're considering small perturbations, this implies that terms multiplied with $\tau'$ should be set to zero in \Eqref{eq:M3:theory:perturbation_ODEs}. This is referred to as the \textit{Tight coupling regime}. In this regime, only the monopole and dipole will be relevant quantities. However, once the higher order multipoles become relevant, we use their lowest non-zero contribution to ensure numerical stability when solving the full ODE. Inserting the initial conditions into \Eqref{eq:M3:theory:perturbation_ODEs} and expanding yields 
\begin{subequations} \label{eq:M3:theory:IC_Thetas}
    \begin{align}
        \Theta_2 &= - \frac{20k}{45\H\tau'}\Theta_1, \label{eq:M3:theory:IC_Theta2} \\
        \Theta_\ell &= - \frac{\ell}{2\ell+1} \frac{k}{\H\tau'}\Theta_{\ell-1} \label{eq:M3:theory:IC_Theta_ell}. 
    \end{align}
\end{subequations}
These are the expressions we will use to compute the multipoles during tight coupling. During tight coupling, photons and electrons behave as a single fluid, which implies that $\Theta_1\simeq-\vb/3$. To avoid numerical instability when multiplying this with $\tau'$ in \Eqref{eq:M3:theory:perturbation_ODEs_Theta1} we will derive a modified expression to compute $\Theta_1'$ and $\vb'$ in \secref{sssec:M3:implementations:tight_coupling_regime}, where we also discuss explain how we determine when to stop using the tight coupling approximation.  


\subsubsection{Line-of-sight integration}\label{sssec:M3:theory:line_of_sight_integration}
From \Eqref{eq:M3:theory:perturbation_ODEs_Theta_ell} we see that we have to solve a huge system of ODEs if we want to probe the CMB to scales around $\ell\sim1000$. However, by exploiting the LOS-integration technique \pnote{Cite}, $\Theta_\ell$ can be obtained from computing the integral 
\begin{equation} \label{eq:M3:theory:Theta_ell_LOS_integration}
    \Theta_\ell(k,x=0) = \int_{-\infty}^0 \tilde{S}(k,x) j_\ell [k(\eta_0 - \eta(x))]\,\dd x, 
\end{equation}
where $\tilde{S}(k,x)$ is the source function, which is given by 
\begin{equation} \label{eq:M3:theory:source_function_LOS}
    \begin{split}
        \tilde{S}(k,x) =& \g \bclosed{\Theta_0 + \Psi + \frac{1}{4} \Pi} + e^{-\tau}\bclosed{\Psi' - \Phi'} \\
        & - \frac{1}{k}\dv{x}(\H \g \vb) + \frac{3}{4k^2}\dv{x}\bclosed{\H \dv{x}(\H\g\Pi)}.
    \end{split}
\end{equation}
Since we have $\Pi=\Theta_2$, the highest photon multipole required to compute any multipole $\Theta_\ell$ from \Eqref{eq:M3:theory:Theta_ell_LOS_integration} is $\Theta_2$. Thus, the number of multipoles we have to compute from the coupled ODEs in \Eqref{eq:M3:theory:perturbation_ODEs} is governed by the value of $\ell_\mathrm{max}$ which ensures sufficient accuracy in $\Theta_2$. 