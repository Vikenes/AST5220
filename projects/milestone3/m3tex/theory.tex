\subsection{Theory}\label{ssec:M3:theory}
In this section we present the equations governing the evolution of the perturbed quantities. We will only present the relevant equations here, which are all obtained from \cite{Dodelson} and \cite{callin}, unless otherwise stated. We refer to \citeauthor{Dodelson} for a detailed derivation of the equations. For the notation, we adapt that of \citeauthor{callin}. Throughout this section, we will consider perturbations in Fourier space.  

\subsubsection{Perturbation quantities}\label{sssec:M3:theory:perturbations}
For the perturbed metric we consider the Newtonian gauge, and write it as  
\begin{equation} \label{eq:M3:theory:metric_perturbation}
    g_{\mu\nu} = 
    \begin{pmatrix}
        -(1+2\Psi) & 0 \\
        0 & a^2 \delta_{ij}(1+2\Phi)
    \end{pmatrix},
\end{equation}
where we have introduced the scalar perturbations $\Psi$ and $\Phi$, which are functions of both position and time. For $\Psi=\Phi=0$ we obtain the FLRW metric. Photon perturbations are defined in terms of the relative temperature variations, $\Theta\equiv \delta T/T$, via 
\begin{equation} \label{eq:M3:theory:temperature_fluctuations}
    T(\vec{k},\mu,\eta) = \zerothorder{T}(\eta)\bclosed{1 + \Theta(\vec{k},\mu,\eta)},
\end{equation}  
where $\vec{k}$ is the Fourier transformed variable corresponding to position $\vec{x}$, and $\mu\equiv \frac{\vec{k}\cdot\vec{p}}{kp}$. The temperature perturbations only depend on the photon momentum in terms of their directions, and we therefore expand $\Theta$ in terms of multipoles as 
\begin{equation} \label{eq:M3:theory:theta_ell_multipolse_expansion}
    \Theta_\ell = \frac{i^\ell}{2}\int_{-1}^1 \pl(\mu)\Theta(\mu)\,\dd\mu \Leftrightarrow \Theta(\mu) = \sum_{\ell=0}^{\infty}\frac{2\ell+1}{i^\ell}\Theta_\ell \pl(\mu),
\end{equation}
where $\pl(\mu)$ are Legendre polynomials. The first three multipoles are the monopole, $\Theta_0$, the dipole, $\Theta_1$ and the quadrupole, $\Theta_2$. The monopole represents the background temperature of the CMB and the dipole represents our relative velocity with respect to the CMB. 

We denote density perturbations of CDM and baryons as $\dcdm$ and $\db$, and their velocity perturbations as $\vcdm$ and $\vb$. From the perturbed Einstein equations one can show that $\dg=4\Theta_0$ and $\vg=-3\Theta_1$ for the photons. 

\subsubsection{Perturbation equations} \label{sssec:M3:theory:perturbation_equations}
The system of ODEs we will solve for the evolution of the perturbed quantities are given in \cite[Eq. (22)]{callin}, where we neglect neutrinos and polarization and set $\Theta_P=\mathcal{N}=0$.   
\begin{subequations} \label{eq:M3:theory:perturbation_ODEs}
    \begin{align}
        \Theta_0' =& - \koverhp\Theta_1 - \Phi', \label{eq:M3:theory:perturbation_ODEs_Theta0}\\
        \Theta_1' =& \frac{k}{3\H}\Theta_0 - \frac{2k}{3\H}\Theta_2 + \frac{k}{3\H}\Psi + \tau' \bclosed{\Theta_1 + \frac{1}{3}v_b}, \label{eq:M3:theory:perturbation_ODEs_Theta1} \\ 
        \Theta_\ell' =& \frac{\ell k}{(2\ell+1)\H}\Theta_{\ell-1} - \frac{(\ell+1)k}{(2\ell+1)\H}\Theta_{\ell+1} \nonumber \\
        & + \tau'\bclosed{\Theta_\ell - \frac{1}{10}\Pi \delta_{\ell,2}},\quad 2\leq\ell<\ell_\mathrm{max}, \label{eq:M3:theory:perturbation_ODEs_Theta_ell} \\
        \Theta_\ell' =& \frac{k}{\H}\Theta_{\ell-1} - c \frac{(\ell+1)}{\H\eta(x)}\Theta_{\ell} + \tau' \Theta_\ell,\quad \ell=\ell_\mathrm{max}, \label{eq:M3:theory:perturbation_ODEs_Theta_ell_max} \\
        \dcdm' =& \frac{k}{\H} \vcdm - 3\Phi', \label{eq:M3:theory:perturbation_ODEs_delta_cdm} \\
        \vcdm' =& - \vcdm \frac{k}{\H} \Psi, \label{eq:M3:theory:perturbation_ODEs_v_cdm} \\
        \db' =& \koverhp \vb - 3\Phi', \label{eq:M3:theory:perturbation_ODEs_delta_b} \\
        \vb' =& -\vb - \koverhp \Psi + \tau' R(3\Theta_1 + \vb), \label{eq:M3:theory:perturbation_ODEs_v_b} \\
        \Phi' =& -\Psi - \frac{k^2}{3\H^2}\Phi + \frac{H_0^2}{2\H^2} \begin{aligned}[t]
            [&\ocdmn a^{-1}\dcdm + \obn a^{-1} \db \\
            &+4\ogn a^{-2} \Theta_0], 
            \end{aligned} \label{eq:M3:theory:perturbation_ODEs_Phi} \\
        \Psi &= -\Phi - \frac{12H_0^2}{k^2 a^2} \ogn \Theta_2. \label{eq:M3:theory:perturbation_ODEs_Psi}
    \end{align}
\end{subequations}
We have $\Pi=\Theta_2$ due to $\Theta_P=0$, and have written $\Pi$ for clarity. The equation for $\Psi$ is an algebraic equation, and its value is inserted in the other equations. Noting that $\Phi'$ is the only derivative of a perturbed quantity found on the RHS of Eqs. \eqref{eq:M3:theory:perturbation_ODEs}, we solve for it first, and insert this value into the other equations. 

In \secref{sec:M4} we discuss the main topic of this report, which is to compute the power spectrum, $C_\ell$, which involves integrating over $\Theta_\ell$. Since we are interested in probing the power spectrum at scales where $\ell>1000$, we need a vast number of photon multipoles. From the ODE, we see that $\Theta_\ell'$ is recursive, and computing $\Theta_\ell$ directly from the ODEs will thus require us to solve thousands of ODEs. This is a highly tedious and computationally expensive process. However, as we will explain in \secref{sssec:M4:theory:line_of_sight_integration}, by exploiting the line-of-sight (LOS) integration technique \cite{LOS_integration}, we can compute any order of $\Theta_\ell$ for $\ell>2$ from $\Theta_0$ and $\Theta_2$. Thus, we only need to use the ODEs to accurately compute $\Theta_0$ and $\Theta_2$. For this, we need to truncate the series at some point. Truncating the series by setting $\Theta_{\ell_\mathrm{max}}=0$ would lead to numerical errors propagating down to $\Theta_2$. \Eqref{eq:M3:theory:perturbation_ODEs_Theta_ell_max} comes from assuming $\Theta_\ell(k,\eta)\sim j_\ell(k\eta)$ at high $\ell$, where $j_\ell(k\eta)$ is the spherical Bessel function, and using the recurrence relation for $j_\ell$ \citep[see][Eq. (46)-(49)]{callin}. By an appropriate choice of $\ell_\mathrm{max}$, we greatly reduce the number of ODEs needed to accurately compute $C_\ell$. 

At early times, the optical depth is very high, since the mean free path of photons is much smaller than the horizon sizes we want to consider. In this regime, only the monopole and dipole are relevant, as higher order multipoles are heavily suppressed \cite[Eq. (9.12)]{Dodelson}. With momentum and pressure as the only relevant parameters to describe the photons, they essentially behave as a fluid. Their frequent collisions with electrons imply that the electrons and photons behave as one fluid. This is referred to as the \textit{tight coupling regime}. We have a large value of $\abs{\tau'}$ during tight coupling, so the photons scatter frequently over the course of a Hubble time, $H^{-1}$. On small scales, this implies that temperature gradients are washed out, since photons reaching an observer from a distance of approximately one mean free path, may originate from other regions in the plasma after several scatterings events have taken place. Hence, electrons are only affected by the nearest photons, and the system is therefore in thermodynamic equilibrium. The implications of the tight coupling regime results in the ODEs in \Eqref{eq:M3:theory:perturbation_ODEs} to become numerically unstable, and we have to resort to approximations. This is discussed in the next section.    

\subsubsection{Initial conditions} \label{sssec:M3:theory:initial_conditions}
We will use adiabatic initial conditions, and follow the convention used by \cite{winther}, with $f_\nu=0$ as we omit neutrinos\footnote{Note that Winther has a sign error in his expression for the initial condition for $\Psi$}. The set of ODEs we solve are linear, so we can choose a particular normalization, which we adjust for when computing the Power spectrum later on. The initial conditions are      
\begin{subequations} \label{eq:M3:theory:pert_ODE_IC}
    \begin{align}
        \Psi &= -\frac{2}{3}, \label{eq:M3:theory:pert_ODE_IC_Psi} \\
        \Phi &= - \Psi, \label{eq:M3:theory:pert_ODE_IC_Phi} \\
        \dcdm &= \db = -\frac{3}{2}\Psi, \label{eq:M3:theory:pert_ODE_IC_delta_cdm} \\
        \vcdm &= \vb = -\frac{k}{2\H} \Psi, \label{eq:M3:theory:pert_ODE_IC_v_cdm} \\
        \Theta_0 &= -\frac{1}{2}\Psi, \label{eq:M3:theory:pert_ODE_IC_Theta0} \\
        \Theta_1 &= \frac{k}{6\H}\Psi. \label{eq:M3:theory:pert_ODE_IC_Theta1} 
    \end{align}
\end{subequations}
The large value of $\tau'$ during tight coupling means that the terms proportional to $\tau'$ in \Eqref{eq:M3:theory:perturbation_ODEs} can be neglected, implying that $\Theta_\ell=0$ for $\ell\geq2$. However, we need the lowest non-zero contribution from these multipoles when numerically integrating the system of ODEs. By expanding in terms of $k/\H\tau'\ll1$ and using the initial conditions, one can obtain the following expression for the higher order multipoles
\begin{subequations} \label{eq:M3:theory:IC_Thetas}
    \begin{align}
        \Theta_2 &= - \frac{20k}{45\H\tau'}\Theta_1, \label{eq:M3:theory:IC_Theta2} \\
        \Theta_\ell &= - \frac{\ell}{2\ell+1} \frac{k}{\H\tau'}\Theta_{\ell-1} \label{eq:M3:theory:IC_Theta_ell}. 
    \end{align}
\end{subequations}
These are the expressions we will use to compute the multipoles during tight coupling. Note that we $\Theta_1\simeq-\vb/3$ in this regime. To avoid numerical instability when multiplying this with $\tau'$ in \Eqref{eq:M3:theory:perturbation_ODEs_Theta1} we derive a modified expression for $\Theta_1'$ and $\vb'$ in \secref{sssec:M3:implementations:tight_coupling_regime}. Furthermore, we discuss in that section how we determine when to stop using the tight coupling approximation. 


\subsubsection{Evolution of different modes}\label{sssec:M3:theory:evolution_of_different_modes}
The evolution of a single mode is governed by \Eqref{eq:M3:theory:perturbation_ODEs}. As the horizon grows, causal physics affects perturbations on increasingly larger scales. Initially we have $k\eta\ll1$, and all modes are outside the horizon. As $\eta$ grows, more and more modes enter the horizon. The epoch in which a mode enters the horizon affects its evolution. To study the evolution of perturbations, we focus on three different modes: one short-wavelength mode that enters the horizon during radiation domination, one large-wavelength mode that enters the horizon during matter domination, and one mode of intermediate scale. 

Defining $x_\mathrm{entry}$ as $k\eta(\xenter)=1$, we choose the wavenumbers
\begin{equation} \label{eq:M3:theory:k_values_for_plotting}
    \begin{split}
        &k\in\Bclosed{0.001,\,0.03,\,0.3}/\mathrm{Mpc}, \\
        &\xenter \in \Bclosed{-5.182,\,-9.726,\,-12.098},
    \end{split}
\end{equation}
where the first and last element of $\xenter$ correspond to modes entering the horizon when $\om\approx0.95$ and $\og\approx0.95$, respectively. The intermediate mode will enter when $\om\approx0.25$ and $\og\approx0.25$. We postpone the details regarding evolution of these modes to \secref{ssec:M3:results}. 

