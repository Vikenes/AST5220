\subsection{Implementation details}\label{ssec:M3:implementations} 
\subsubsection{Tight coupling regime} \label{sssec:M3:implementations:tight_coupling_regime}
During tight coupling, \Eqref{eq:M3:implementations:TC_ODEs_Theta_1} needs to be modified to avoid numerically instability at early times. Following \citeauthor{callin}, the tight coupling regime is computed with 
\begin{subequations} \label{eq:M3:implementations:TC_ODEs}
    \begin{align}
        \varrho q =& -[(1-R)\tau' + (1+R)\tau''](3\Theta_1 + \vb) \nonumber \\
            & -\koverhp\Psi + (1-\frac{\H'}{\H})\koverhp(-\Theta_0 + 2\Theta_2) - \koverhp\Theta_0', \label{eq:M3:implementations:TC_ODEs_q} \\
        \vb' =& \frac{1}{1+R}\bclosed{-\vb-\koverhp\Psi + R(q+\koverhp[-\Theta_0+2\Theta_2-\Psi])}, \label{eq:M3:implementations:TC_ODEs_v_b} \\
        \Theta_1' =& \frac{1}{3}(q-\vb'), \label{eq:M3:implementations:TC_ODEs_Theta_1} 
    \end{align}
\end{subequations} 
where we have introduced the parameter 
\begin{equation} \label{eq:M3:implementations:q_prefactor}
    \varrho = (1+R)\tau' + \frac{\H'}{\H} - 1.
\end{equation}
In the tight coupling regime we use the initial conditions given in \Eqref{eq:M3:theory:pert_ODE_IC}. We then solve the ODEs given in \Eqref{eq:M3:theory:perturbation_ODEs}, but compute $\Theta_1'$ and $v_b'$ in the end by using \Eqref{eq:M3:implementations:TC_ODEs}. In this regime we use Eqs. \Eqref{eq:M3:theory:IC_Theta2} and \eqref{eq:M3:theory:IC_Theta_ell} for $\Theta_2$ and $\Theta_\ell$, respectively. 

Next, we need to determine when the tight coupling approximation is valid. We use the same conditions as \citeauthor{callin}, namely that the approximation holds as long as $\abs{\tau'}>10$ and $\abs{k/\H\tau'}<1/10$, and that it should not be used after the start of recombination. For the latter condition, we choose $x<-8.3$ as the time when recombination starts. If any of these three conditions fails, we switch to the full ODEs. We then use the final results from the tight coupling regime as initial conditions for the ODEs in \Eqref{eq:M3:theory:perturbation_ODEs}, and integrate up to $x=0$. 

\subsubsection{Integration of ODEs} \label{sssec:M3:implementations:integration_details}

When solving the perturbation ODEs we integrate across $x$ for each mode $k$. We choose $N_k=200$ logarithmically spaced values of $k\in[k_\mathrm{min},\,k_\mathrm{max}]$, where $k_\mathrm{min}=5\cdot10^{-5}/\mathrm{Mpc}$ and $k_\mathrm{max}=0.3/\mathrm{Mpc}$. For the photon multipoles, we set $\ell_\mathrm{max}=7$.

Once the ODEs have been solved for all combinations of $x$ and $k$, we make splines of the integrated perturbation quantities, except for $\Theta_\ell$ for $\ell>2$, as this will be solved using the LOS integration. 