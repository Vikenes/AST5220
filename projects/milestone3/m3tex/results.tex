\subsection{Results}\label{ssec:M3:results}
In this section we present plots showing the evolution of some perturbation quantities. In Figs. \ref{fig:M3:results:deltas}-\ref{fig:M3:results:Theta2} we indicate the time when tight coupling end by a dotted black vertical line. The modes we have chosen to include in these results are $k=\Bclosed{0.001\mathrm{Mpc},0.01\mathrm{Mpc},0.1\mathrm{Mpc}}$.

%===== overdensities =========
\begin{figure}[ht!]
    \includefig{deltas}
    \caption{Time evolution of $\dcdm$ (solid line) and $\db$ (dashed line) for three different modes. Long before and long after recombination, the two species follow each other closely. Around the epoch of recombination we see large fluctuations for the baryons, particularly for the highest k-mode, i.e. the smallest scales.}
    \label{fig:M3:results:deltas}
\end{figure}

\begin{figure}[ht!]
    \includefig{delta_gamma}
    \caption{Time evolution of $\dg$ for three different modes. }
    \label{fig:M3:results:delta_gamma}
\end{figure}


\begin{figure}[ht!]
    \includefig{vels}
    \caption{Evolution of $\vcdm$ (solid line) and $\vb$ (dashed line) for three different modes. The most evident deviation between the baryons and CDM is for the largest mode.}
    \label{fig:M3:results:vels}
\end{figure}

\begin{figure}[ht!]
    \includefig{v_gamma}
    \caption{Evolution of $\vg$.}
    \label{fig:M3:results:v_gamma}
\end{figure}



\begin{figure}[ht!]
    \includefig{Phi}
    \caption{Evolution of $\Phi$.}
    \label{fig:M3:results:Phi}
\end{figure}



\begin{figure}[ht!]
    \includefig{Phi_plus_Psi}
    \caption{Evolution of the combined metric perturbations, $\Phi+\Psi$}
    \label{fig:M3:results:Phi_plus_Psi}
\end{figure}



\begin{figure}[ht!]
    \includefig{Theta2}
    \caption{Evolution of the quadrupole, $\Theta_2$.}
    \label{fig:M3:results:Theta2}
\end{figure}
