\subsection{Results}\label{ssec:M3:results}

In this section we present plots showing the evolution of some perturbation quantities. For the monopole and dipole, we express plot these in terms of $\dg$ and $\vg$, respectively. For easier comparison between evolution of baryons, photons and CDM, we plot CDM and baryons together in one figure, and baryons and photons together in another figure. In both cases we include all three modes in the same figure. This approach prevents cluttering, and helps us understand the interaction between the three constituents, as baryons interact with both CDM and photons. 

In Figs. \ref{fig:M3:results:deltas}-\ref{fig:M3:results:Theta2} we indicate the time of matter-radiation equality by a dotted black vertical line. We stress that matter-radiation equality occurs at $x_\mathrm{eq}=-8.6577$, as opposed to the time given in \tabref{tab:M1:results:time_values}, since $\neff=0$ now. The time of decoupling and recombination, however, is less affected, occurring at $x\sim-6.98$ as before. We also mark the horizon entry time for each mode, given in \Eqref{eq:M3:theory:k_values_for_plotting}, shown as dash-dotted lines. For brevity, we will refer to modes with large $k$ as \textit{short modes} and modes with small $k$ as \textit{long modes}, as they are related to small and large scales, respectively. 

We begin by studying the density evolutions in \secref{sssec:M3:results:density_perturbations}, and spend a considerable time interpreting these results qualitatively. This discussion provides a simpler way of understanding the mechanism behind the remaining perturbation quantities.

%===== overdensities =========
\subsubsection{Density perturbations} \label{sssec:M3:results:density_perturbations}
The absolute value of the density evolution of baryons, CDM and photons is shown in \figref{fig:M3:results:deltas}, where the upper panel shows $\dcdm$ and $\db$ and the lower panel shows $\db$ and $\dg$. Before any of the modes enter the horizon, the baryons and CDM behave identically. For the two shorter modes, $\db$ begins deviating noticeably from $\dcdm$ as upon entering the horizon. For the CDM, the growth of the density perturbation is due to CDM clustering as they undergo gravitational collapse, taking place around the time when the modes enter the horizon. This attracts baryons, which fall into the CDM potential wells. As baryons collapse, pressure from photons increases and pushes them outwards again. This phenomenon is apparent in the lower panel, where $\db$ and $\dg$ have the same evolution before decoupling takes places at $x\sim-7$. This process repeats for the two shorter modes, which have entered the horizon before recombination begins. When decoupling takes place, the photon pressure is unable to prevent gravitational collapse, and baryons fall into the gravitational wells set up by CDM. Consequently, $\db$ follows $\dcdm$ at late times. The photons can move freely after decoupling, and their temperature remains approximately unaffected towards $x=0$. Having oscillations in time Fourier space, this corresponds to moving plane waves in real space, which is why wee see some fluctuations in $\dg$ at late times. There is also noticeable in $\dg$ at $x\sim0$ for the shortest mode. This is a result of accelerated expansion of the Universe at late times, which we discuss further in \secref{sssec:M3:results:potential_perturbations}     

In general, we see more oscillations in $\db$ and $\dg$ for smaller modes. The large mode enters the horizon long after decoupling, and the baryon overdensity is unaffected by photons when collapse begins. Neglecting terms with factors of $k$ in \Eqref{eq:M3:theory:perturbation_ODEs}, we get $\db=\dcdm$ at all times, since they have identical initial conditions. 

The value of $\dg$ after decoupling differs between the three modes, where shorter modes have a lower final temperature. Decoupled photons are redshifted as they move out of gravitational potential wells. On small scales, where CDM have more time to cluster before decoupling, the escaping photons become more redshifted. Additionally, the actual evolution of the potential wells before decoupling is an important factor in the final photon temperatures. For the shortest mode, we see that $\dcdm$ grows differently right after horizon entry, compared to after $\xeq$. This may be a result of matter density starting to affect the expansion rate of the Universe. We will discuss the effect of expansion in more detail in \secref{sssec:M3:results:metric_perturbations}. We see an example of how expansion affects the overdensities, as there is a slight decrease in $\dg$ near $x=0$ with $\dcdm$ and $\db$ flattening out. This occurs when the Universe becomes dominated by dark energy, where accelerated expansion is more effective than the accretion of matter. 
\begin{figure}[ht!]
    \includefig{deltas\kmodesplot}
    \includefig{delta_baryon_photon\kmodesplot} 
    \caption{Evolution of $\delta$ for modes with three different wavenumbers. Upper panel: Comparing $\dcdm$ (solid line) and $\db$ (dashed line). Lower panel: Comparing $\db$ (solid line) and $\dg=4\Theta_0$ (dashed line).}
    \label{fig:M3:results:deltas}
\end{figure}


\subsubsection{Velocity perturbations}\label{sssec:M3:results:velocity_perturbations}
The evolution of the velocities is shown in \figref{fig:M3:results:vels}, where we compare baryons and CDM in the upper panel, and baryons and photons in the lower panel. Note that we have used a linear scale for the lower panel to reduce clutter.   

In \figref{fig:M3:results:deltas}, the oscillatory behaviour of baryons for short modes is also apparent in their velocities, where the velocity extrema coincide with density minima, as expected. Similar behaviour is observed for CDM velocities, with both baryons and CDM coinciding at early and late times, and decoupling effects in the middle for the two shorter modes. For the longest mode, $\vcdm$ and $\vb$ follow the same argument as their densities, resulting in $\vcdm=\vb$. 

For the shortest mode, $\vcdm$ is no longer a strictly increasing function at early times. Around $x=-10$, $\vcdm$ declines and becomes smaller than the intermediate mode around $\xeq$\footnote{The reason for this behaviour has confused great minds for several light weeks. Their best estimate to this day is that it's expansion related}. This shift persists all the way towards $x=0$. During tight coupling we have $\vb\sim\vg$, and near decoupling, at $x\sim-7$, we see that the photons decouple from the plasma. After that, the photons undergo damped oscillations, with a higher frequency for shorter modes. 

\begin{figure}[ht!]
    \includefig{vels\kmodesplot}
    \includefig{vel_baryon_photon\kmodesplot}
    \caption{Evolution of velocities for three different wavenumbers. Upper panel: $\vcdm$ (solid line) compared to $\vb$ (dashed line). Lower panel: $\vb$ (solid line) compared to $\vg$ (dashed line).}
    \label{fig:M3:results:vels}
\end{figure}


\subsubsection{Potential perturbations} \label{sssec:M3:results:potential_perturbations}
The evolution of $\Phi$ is illustrated in \figref{fig:M3:results:Phi}. This analysis is based on the discussion presented in \cite[Ch. 8]{Dodelson}. During both matter and radiation domination, $\Phi$ can be well-approximated as being constant, with a transition period in between. As the shortest mode enters the horizon when $x\ll\xeq$, the potential decays and begins to oscillate, leading to the damped oscillations observed for photons and baryons in \figref{fig:M3:results:deltas}. During matter domination, for a mode that has entered the horizon, we get a constant $\Phi$, as the matter accretion is balanced by the Universe's expansion. The constancy of $\Phi$ explains why all modes in \figref{fig:M3:results:deltas} grow at the same rate at late times, long after decoupling. 

From the transition period mentioned earlier, we can compare the changes in the growth of $\dcdm$ at points where $\Phi$ exhibits irregular behaviour. Most notably, the minima in $\Phi$ at around $x\sim-10$ roughly corresponds to the change in $\dcdm$ for the short mode at the same $x$.

\begin{figure}[ht!]
    \includefig{Phi\kmodesplot}
    \caption{Evolution of $\Phi$.}
    \label{fig:M3:results:Phi}
\end{figure}

The sum of the two metric potentials, $\Phi+\Psi$, is plotted in \figref{fig:M3:results:Phi_plus_Psi}. This quantity corresponds to the second term on the right-hand side of \Eqref{eq:M3:theory:perturbation_ODEs_Psi} and is proportional to the quadrupole moment, $\Theta_2$, which is depicted in \figref{fig:M3:results:Theta2}. Specifically, we have $\Phi+\Psi \propto -\frac{\Theta_2}{k^2} e^{-2x}$. Since $\Theta_2(x\ll\xeq)\approx 0$, the sum of the potentials is initially zero. Although the proportionality constant is very small, for larger $x$, late in the matter dominated era, there is a narrow range where both $\Theta_2$ contributes and the exponent is large enough to give a non-zero contribution.

For the evolution of $\Theta_2$, we first begin to see growing oscillations in the short mode around $x=\xeq$. This growth is abruptly suppressed once decoupling happens, where we also had an abrupt decrease of $\vg$ in \figref{fig:M3:results:vels}. Similar behaviour is seen for the remaining two modes. For these modes, oscillations start around the time of decoupling, and continues with an amplitude related to the amplitude of the oscillations of $\vg$. The largest mode is also responsible for the largest value of $\Theta_2$ at $x=0$. 

The evolution of $\Theta_2$ is characterized by growing oscillations in the short mode around $x=\xeq$, which are suddenly suppressed upon decoupling. This abrupt decrease is also seen as a decrease in $\vg$ in \figref{fig:M3:results:vels}. Similar behaviour is also observed for the remaining two modes. Oscillations for these modes start around decoupling and persist with an amplitude that is related to the amplitude of $\vg$ oscillations. The largest mode exhibits the highest $\Theta_2$ value at $x=0$.

\begin{figure}[ht!]
    \includefig{Phi_plus_Psi\kmodesplot}
    \caption{Evolution of the combined metric perturbations, $\Phi+\Psi$}
    \label{fig:M3:results:Phi_plus_Psi}
\end{figure}

  
\begin{figure}[ht!]
    \includefig{Theta2\kmodesplot}
    \caption{Evolution of the quadrupole, $\Theta_2$. At $x<-10$, there is no apparent contribution of this quantity.}
    \label{fig:M3:results:Theta2}
\end{figure}
