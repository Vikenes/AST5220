\subsection{Results}\label{ssec:M3:results}

In this section we present plots showing the evolution of some perturbation quantities. For the monopole and dipole, we express plot these in terms of $\dg$ and $\vg$, respectively. For easier comparison between evolution of baryons, photons and CDM, we plot CDM and baryons together in one figure, and baryons and photons together in another figure. In both cases we include all three modes in the same figure. This prevents cluttering of the figures, and helps understand the physics behind the three constituents, as baryons interact with both CDM and photons. 

In Figs. \ref{fig:M3:results:deltas}-\ref{fig:M3:results:Theta2} we indicate the time of matter-radiation equality by a dotted black vertical line. Since we have set $\neff=0$, the matter-radiation equality we consider here occurs at $x_\mathrm{eq}=-8.6577$, as opposed to the time given in \tabref{tab:M1:results:time_values}. For brevity, we will from now on refer to modes with large $k$ as \textit{short modes} and modes with small $k$ as \textit{long modes}, as they are related to small and large scales, respectively. 

We begin by studying the density evolutions in \secref{sssec:M3:results:density_perturbations}, and spend a considerable time discussing these results in detail. This helps us understand the underlying physics behind the perturbations, and allows for easier interpretation of the remaining results. 

%===== overdensities =========
\subsubsection{Density perturbations} \label{sssec:M3:results:density_perturbations}
We begin by plotting the density evolution for the matter components and photons, shown in \figref{fig:M3:results:deltas}, where the upper panel shows $\dcdm$ and $\db$ and the lower panel shows $\db$ and $\dg$. We see that the baryons and CDM follow each other closely during the tight coupling regime, and that the $\db$ deviates noticeably from $\dcdm$ during recombination for the shortest mode. On small scales we will initially have overdense regions of CDM undergoing gravitational collapse. This will attract baryons which will fall into the CDM potential wells. As baryons collapse, pressure from photons will increase and push the baryons outwards again. We see a manifestation of this process in the lower panel, where $\db$ and $\dg$ follow each other as a single fluid before decoupling. 

This process will repeat until recombination occurs, after which the photons move freely, and baryons can collapse into the gravitational wells set up by CDM. This causes $\db$ to follow $\dcdm$ at late times. When photons decouple from matter and move freely, their temperature are no longer affected by interactions, and we get an approximately constant $\dg$ towards $x=0$.  

For the intermediate mode, oscillations starts much closer to the time of decoupling, and we therefore only see few oscillations in both panels. The large mode enters the horizon much later, with gravitational collapse taking place long after decoupling. Thus, we get $\db\sim\dcdm$ at all times, with a decoupling of $\dg$ taking place once recombination has affected the mode on this scale \pnote{Fix last sentence, explain onset of photon decoupling}. 

The value of $\dg$ after decoupling differs between the three modes, where shorter modes have a lower final value of $\dg$, i.e. a lower temperature. The gravitational wells are dominated by CDM before decoupling. On small scales CDM will have more time to cluster before decoupling, and is the reason for why we have $\dcdm$ is larger for at a given $x$ for shorter modes. When photons decouple they are redshifted as they climb out of gravitational wells, causing a reduction in their temperature. This redshift is stronger for short modes where $\dcdm$ is higher, and explains the different values of $\dg$ we see at late times for the three modes. 

There are two main factors governing the onset of decoupling for different modes.  

\begin{figure}[ht!]
    \includefig{deltas\kmodesplot}
    \includefig{delta_baryon_photon\kmodesplot} 
    \caption{Upper panel: Time evolution of $\dcdm$ (solid line) and $\db$ (dashed line) for three different modes. Lower panel: Evolution of $\dg=4\Theta_0$}
    \label{fig:M3:results:deltas}
\end{figure}


\subsubsection{Velocity perturbations}\label{sssec:M3:results:velocity_perturbations}
For the velocities, we see in \figref{fig:M3:results:vel_CDM_b} similar behaviour between CDM and baryons as we did for the densities. 

\begin{figure}[ht!]
    \includefig{vels\kmodesplot}
    \includefig{vel_baryon_photon\kmodesplot}
    \caption{Evolution of $\vcdm$ (solid line) and $\vb$ (dashed line) for three different modes.}
    \label{fig:M3:results:vel_CDM_b}
\end{figure}

In \note{figure \text{vel\_baryon\_photon\_k\_001\_01\_1} (I removed it)} we plot $\vb$ and $\vg$ using a linear scale. At early times, during the tight coupling regime, we have $\vb=\vg$. We see that decoupling between photons and baryons is evident for all three modes of consideration here.  


% \begin{figure}[ht!]
%     \includefig{vel_baryon_photon_k_001_01_1}
%     \caption{Evolution of $\vb$ (solid line) and $\vg$ (dashed line) for three different modes.}
%     \label{fig:M3:results:vel_photon_b}
% \end{figure}

\subsubsection{Potential perturbations} \label{sssec:M3:results:metric_perturbations}
The evolution of $\Phi$ is shown in \figref{fig:M3:results:Phi}. The short modes enter the horizon during radiation domination, where the potential begins to decay and oscillate, as expected \pnote{cite}. For all three modes we have $\Phi$ approximately constant during matter domination, as predicted \pnote{cite}. The decay we see towards the end for all three modes are due to dark energy beginning to dominate. 
\begin{figure}[ht!]
    \includefig{Phi\kmodesplot}
    \caption{Evolution of $\Phi$.}
    \label{fig:M3:results:Phi}
\end{figure}


\begin{figure}[ht!]
    \includefig{Phi_plus_Psi\kmodesplot}
    \caption{Evolution of the combined metric perturbations, $\Phi+\Psi$}
    \label{fig:M3:results:Phi_plus_Psi}
\end{figure}



\begin{figure}[ht!]
    \includefig{Theta2\kmodesplot}
    \caption{Evolution of the quadrupole, $\Theta_2$. At $x<-10$, there is no apparent contribution of this quantity.}
    \label{fig:M3:results:Theta2}
\end{figure}
