\subsection{Results}\label{ssec:M2:results}


\subsubsection{Electron fraction} \label{sssec:M2:results:electron_fraction}
The evolution of the electron fraction with time is shown in \figref{fig:M2:results:recombination_compare_Xe_peebles_saha}. Here, we have included the solution obtained through the Saha equation alone for comparison. In the figure we also show the time periods when recombination occurs according to the two solutions, i.e. when $\Xe=0.1$. The time of decoupling has been omitted, since it is very close to the recombination time, as shown in table \ref{tab:M2:results:rec_and_dec_time_table_Peebles}. 

We see that the Peebles equation predicts the production of neutral Hydrogen to take much longer time than the Saha equation. As previously discussed, recombination occurs at a much lower temperature than the binding energy of Hydrogen, even with the Saha approximation, which does not take complicated atomic physics into account. Generally, the main reason for this delay is that the photons do not have a single energy, but rather a distribution of energies. The vast number of photons in the Universe means that at $T\ll\epsn$, there are many photons available to ionize Hydrogen atoms, thereby delaying recombination. Towards $x=0$ the Peebles equation give $\Xe\ll1$, which appears to settle at an approximately fixed value. From our solution we find that the current value is $\Xe(x=0)\approx 2.026\cdot10^{-4}$. On the other hand, the Saha equation yields a rapidly decaying electron fraction, with $\Xe\ll 10^{-4}$ occurring long before $x=0$ is reached.  

\begin{figure}[ht!]
    \includefig{recombination_compare_Xe_peebles_saha}
    \caption{Free electron fraction computed from the Saha equation only (dashed green curve) and from the Peebles equation (solid blue curve), where the Saha equation was used at early times, until $\Xe<0.99$. The recombination times is shown for both solutions.}
    \label{fig:M2:results:recombination_compare_Xe_peebles_saha}
\end{figure}

\subsubsection{Optical depth and visibility function} \label{sssec:M2:results:optical_depth_and_visibility_function}
The optical depth, and its first two derivatives is shown in \figref{fig:M2:results:tau_plot}. We see that the three quantities rapidly drop by several orders of magnitude near $x=-7$, around the time of recombination. Looking at \figref{fig:M2:results:g_plot} we see that the rapidly varying optical depth results in a sharply peaked $\g$. 

\begin{figure}[ht!]
    \includefig{tau_plot}
    \caption{The optical depth, $\tau(x)$ and its two first derivatives with respect to $x$.}
    \label{fig:M2:results:tau_plot}
\end{figure}


\begin{figure}[ht!]
    \includefig{g_plot}
    \caption{The visibility function, $\gx$ (solid curve), and its derivatives, $\g'(x)/10$ (dashed curve) and $\g''(x)/300$ (dotted curve). The derivatives have been scaled in order to view them all in the same plot.}
    \label{fig:M2:results:g_plot}
\end{figure}

\subsubsection{Times of recombination and decoupling} \label{sssec:M2:results:times_of_recombination_and_decoupling}
The period of recombination and decoupling, as computed from the Peebles equation, is given in table \ref{tab:M2:results:rec_and_dec_time_table_Peebles}. We see that the photons and baryons decouple later than recombination occurs, as is expected. This does depend on how we define the two periods, and is thus not a very conclusive result on its own. Using a higher value of $\Xe$ for instance would yield a larger separation between the two events.   

\tablesMilestoneTwo{rec_and_dec_time_table_Peebles.tex}  

\tablesMilestoneTwo{rec_and_dec_time_table_Saha.tex} 

  