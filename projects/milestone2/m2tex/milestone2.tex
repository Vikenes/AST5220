\section{Milestone II}\label{M2}
Having successfully implemented the background cosmology, the next step in 
developing our model is to include interactions between particles. After the Big Bang, the early universe was highly ionized. Due to Thompson scattering, photons were strongly coupled to baryons. As the Universe expanded, and the temperature dropped, neutral atoms were able to form, and the photons were able to escape from the plasma. These are the CMB photons we observe today. The period where neutral atoms formed is called recombination, and will be the main topic of this section. Our goal is to compute the number density of free electrons in the Universe, and use this to estimate when recombination occurred. The evolution of free electrons will affect both structure formation and the resulting power spectrum of the CMB photons, as we will study later.

We will use both the Saha and Peebles equation to compute the electron number density, and use this to compute the optical depth. From the optical depth, we will compute the so-called \textit{visibility function}. In this section we will adapt natural units, where we set $c=\hbar=k_B=1$. Additionally, we will make the assumption that Hydrogen is the only element present in the universe. Hence, we use $Y_p=0$, rather than the value from Planck, given in 
\Eqref{eq:Appendix:Fiducial_cosmology_parameters}, as we neglect Helium and heavier elements. \cite{Dodelson}


The code for this milestone can be found on my GitHub repository: \url{https://github.com/Vikenes/AST5220/tree/main/projects/milestone2}

\subsection{Theory}\label{ssec:M3:theory}
In this section we present the equations governing the evolution of the perturbed quantities. We will only present the relevant equations here, which are all obtained from \cite{Dodelson} and \cite{callin}, unless otherwise stated. We refer to \citeauthor{Dodelson} for a detailed derivation of the equations. For the notation, we adapt that of \citeauthor{callin}. 

\subsubsection{Metric Perturbations}\label{sssec:M3:theory:perturbations}
For the perturbed metric we consider the Newtonian gauge \pnote{Explain/cite}, and write it as  
\begin{equation}
    g_{\mu\nu} = 
    \begin{pmatrix}
        -(1+2\Psi) & 0 \\
        0 & a^2 \delta_{ij}(1+2\Phi)
    \end{pmatrix},
\end{equation}
where we have introduced the scalar perturbations $\Psi$ and $\Phi$, which are functions of both position and time. For $\Psi=\Phi=0$ we obtain the FLRW metric. Photon perturbations are defined in terms of the relative temperature variations, $\Theta$, via 
\begin{equation} \label{eq:M3:theory:temperature_fluctuations}
    T(\vec{k},\mu,\eta) = \zerothorder{T}(\eta)\bclosed{1 + \Theta(\vec{k},\mu,\eta)},
\end{equation}  
where $\vec{k}$ is the Fourier transformed variable corresponding to position $\vec{x}$, and $\mu\equiv \frac{\vec{k}\cdot\vec{p}}{kp}$. $\Theta$ only depends on the photon momentum in terms of their directions, and are expanded in terms of multipoles as 
\begin{equation} \label{eq:M3:theory:theta_ell_multipolse_expansion}
    \Theta_\ell = \frac{i^\ell}{2}\int_{-1}^1 \pl(\mu)\Theta(\mu)\,\dd\mu,\,\quad \Theta(\mu) = \sum_{\ell=0}^{\infty}\frac{2\ell+1}{i^\ell}\Theta_\ell \pl(\mu),
\end{equation}
where $\pl(\mu)$ are Legendre polynomials. \note{Explain Fourier space, and its relation to the CMB.} For CDM we denote the density and velocity perturbations as $\dcdm$ and $\vcdm$, respectively, and similarly denote the baryon perturbations as $\db$ and $\vb$.  

\subsubsection{Perturbation equations}
In Fourier space, the system of ODEs we have to solve are  
\begin{subequations}
    \begin{align}
        \Theta_0' =& - \koverhp\Theta_1 - \Phi', \\
        \Theta_1' =& \frac{k}{3\H}\Theta_0 - \frac{2k}{3\H}\Theta_2 + \frac{k}{3\H}\Psi \bclosed{\Theta_1 + \frac{1}{3}v_b} \\ 
        \Theta_\ell' =& \frac{\ell k}{(2\ell+1)\H}\Theta_{\ell-1} - \frac{(\ell+1)k}{(2\ell+1)\H}\Theta_{\ell+1} \nonumber \\
        & + \tau'\bclosed{\Theta_\ell - \frac{1}{10}\Pi \delta_{\ell,2}},\quad 2\leq\ell<\ell_\mathrm{max}, \\
        \Theta_\ell' =& \frac{k}{\H}\Theta_{\ell-1} - c \frac{(\ell+1)}{\H\eta(x)}\Theta_{\ell} + \tau' \Theta_\ell,\quad \ell=\ell_\mathrm{max}, \\
        \dcdm' =& \frac{k}{\H} \vcdm - 3\Phi', \\
        \vcdm' =& - \vcdm \frac{k}{\H} \Psi, \\
        \db' =& \koverhp \vb - 3\Phi', \\
        \vb' =& -\vb - \koverhp \Psi + \tau' R(3\Theta_1 + \vb), \\
        \Phi' =& -\Psi - \frac{k^2}{3\H^2}\Phi + \frac{H_0^2}{2\H^2} \begin{aligned}[t]
            [&\ocdmn a^{-1}\dcdm + \obn a^{-1} \db \\
            &+4\ogn a^{-2} \Theta_0] 
            \end{aligned}, \\
        \Psi &= -\Phi - \frac{12H_0^2}{k^2 a^2} \ogn \Theta_2
    \end{align}
\end{subequations}



The initial conditions are 
\begin{subequations}
    \begin{align}
        \Psi &= -\frac{2}{3}, \\
        \Phi &= - \Psi, \\
        \dcdm &= \db = -\frac{3}{2}\Psi, \\
        \vcdm &= \vb = -\frac{k}{2\H} \Psi, \\
        \Theta_0 &= -\frac{1}{2}\Psi, \\
        \Theta_1 &= \frac{k}{6\H}\Psi, \\
        \Theta_2 &= - \frac{20k}{45\H\tau'}\Theta_1, \\
        \Theta_\ell &= - \frac{\ell}{2\ell+1} \frac{k}{\H\tau'}\Theta_{\ell-1} \pnote{Needed?}.
    \end{align}
\end{subequations}



The tight coupling regime is computed with 
\begin{subequations}
    \begin{align}
        \varrho q =& -[(1-R)\tau' + (1+R)\tau''](3\Theta_1 + \vb) \nonumber \\
            & -\koverhp\Psi + (1-\frac{\H'}{\H})\koverhp(-\Theta_0 + 2\Theta_2) - \koverhp\Theta_0', \\
        \vb' =& \frac{1}{1+R}\bclosed{-\vb-\koverhp\Psi + R(q+\koverhp[-\Theta_0+2\Theta_2-\Psi])}, \\
        \Theta_1' =& \frac{1}{3}(q-\vb'),
    \end{align}
\end{subequations}
where we introduced the parameter 
\begin{equation}
    \varrho = (1+R)\tau' + \frac{\H'}{\H} - 1.
\end{equation}


\subsubsection{Line-of-sight integration}\label{sssec:M3:theory:line_of_sight_integration}

\begin{equation}
    \Theta_\ell(k,x=0) = \int_{-\infty}^0 \tilde{S}(k,x) j_\ell [k(\eta_0 - \eta(x))]\,\dd x, 
\end{equation}
where $\tilde{S}(k,x)$ is the source function 
\begin{equation}
    \begin{split}
        \tilde{S}(k,x) =& \g \bclosed{\Theta_0 + \Psi + \frac{1}{4} \Pi} + e^{-\tau}\bclosed{\Psi' - \Phi'} \\
        & - \frac{1}{k}\dv{x}(\H \g \vb) + \frac{3}{4k^2}\dv{x}\bclosed{\H \dv{x}(\H\g\Pi)}.
    \end{split}
\end{equation}

\subsection{Implementation details}\label{ssec:M3:implementations} 

\subsubsection{Tight coupling regime} \label{sssec:M3:implementations:tight_coupling_regime}
The tight coupling regime is computed with 
\begin{subequations} \label{eq:M3:implementations:TC_ODEs}
    \begin{align}
        \varrho q =& -[(1-R)\tau' + (1+R)\tau''](3\Theta_1 + \vb) \nonumber \\
            & -\koverhp\Psi + (1-\frac{\H'}{\H})\koverhp(-\Theta_0 + 2\Theta_2) - \koverhp\Theta_0', \label{eq:M3:implementations:TC_ODEs_q} \\
        \vb' =& \frac{1}{1+R}\bclosed{-\vb-\koverhp\Psi + R(q+\koverhp[-\Theta_0+2\Theta_2-\Psi])}, \label{eq:M3:implementations:TC_ODEs_v_b} \\
        \Theta_1' =& \frac{1}{3}(q-\vb'), \label{eq:M3:implementations:TC_ODEs_Theta_1} 
    \end{align}
\end{subequations}
where we introduced the parameter 
\begin{equation} \label{eq:M3:implementations:q_prefactor}
    \varrho = (1+R)\tau' + \frac{\H'}{\H} - 1.
\end{equation}




\subsection{Results}\label{ssec:M4:results}
 