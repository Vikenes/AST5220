

\subsection{Theory}\label{ssec:M2:theory}

\subsubsection{Optical depth and visibility function} \label{sssec:M2:optical_depth_and_visibility_function}

A source that emits light with an intensity $I_0$ is attenuated by a factor $e^{-\tau(x)}$ as it travels through a medium, where $\tau$ is the optical depth of the medium. In the early universe we have $\tau\gg1$, and the universe is said to be optically thick, as light is unable to \note{emerge}. As the universe expands and neutral atoms form, the universe becomes optically thin, $\tau\ll1$, and light is able to escape the primordial plasma and reach us. A common way to define when recombination occurs is the time when we have $\tau=1$. Considering Thompson scattering only, the optical depth is defined in terms of the scale factor, $a$, as \cite[Eq. (5)]{callin} 
\begin{equation} \label{eq:M2:theory:optical_depth_integral_definition}
    \tau(\eta) = \int_\eta^{\eta_0} \dd \eta' \,\ne \sigma_T a,
\end{equation}
where $\ne$ is the number density of free electrons,
\begin{equation} \label{eq:M2:theory:thomspon_cross_section}
    \sigma_T=\frac{8\pi\alpha^2}{3 m_e^2} = 6.6524587158\cdot 10^{-29}\unit{m^2} 
\end{equation}
is the Thompson cross-section and $a$ is the scale factor. $\alpha$ is the fine-structure constant and $m_e$ is the electron mass. Using \Eqref{eq:M1:theory:eta_ODE}, we can rewrite \Eqref{eq:M2:theory:optical_depth_integral_definition} as a differential equation 
\begin{equation} \label{eq:M2:theory:tau_ODE}
    \dv{\tau}{x} = - \frac{\ne \sigma_T}{H},
\end{equation}
which we can solve numerically once we know $n_e$. The initial condition is $\tau(x=0)=0$.

From the optical depth, we obtain the so-called \textit{visibility function} \cite[Eq. (8)]{callin} 
\begin{equation} \label{eq:M2:theory:g_tilde_of_x_definition}
    \gx = -\tau' e^{-\tau},
\end{equation}
which is normalized as 
\begin{equation} \label{eq:M2:theory:g_tilde_normalization}
    \int_{-\infty}^0\dd x\,\gx = 1.
\end{equation}
The normalization means that $\gx$ is a probability distribution, and we may interpret it as the probability of an observed CMB photon today having experienced its last scattering at a time $x$. The visibility function is sharply peaked around the time when recombination takes place, and the period of recombination is thus often referred to as the surface of last scattering. We therefore have two ways of estimating when photon baryon decoupling occurred, either as the time when the universe became optically thin, i.e. when $\tau=1$, or as the time when the visibility function is at its maximum value.

\subsubsection{Electron density \note{(working title)}} \label{sssec:M2_electron_density}
The final thing we need to compute $\tau$ and $\g$ is the number density of free electrons. To compute the evolution of free electrons, we have to consider the Boltzmann equation \cite[Eq. (3.19)]{Dodelson}
\begin{equation} \label{eq:M2:theory:Boltzmann_equation}
    \dv{f}{t} = C[f],
\end{equation}
where $f$ is the distribution function and $C[f]$ is a collision term. 

The Boltzmann equation we will consider is \cite[Eq. (4.8)]{Dodelson} \note{(explain assumptions and simplifications.)}
\begin{equation} \label{eq:M2:theory:Boltzmann_equation_four_particles_scale_factor}
    a^{-3}\dv{(n_1 a^3)}{t} = \equilibriumn[1] \equilibriumn[2] \expval{\sigma v} \BBclosed{\frac{n_3 n_4}{\equilibriumn[3] \equilibriumn[4]} - \frac{n_1 n_2}{\equilibriumn[1] \equilibriumn[2]}},
\end{equation}
where $n_i$ denotes the number density of a particle species $i$, and $\equilibriumn[i]$ denotes the number density in equilibrium. We are interested in recombination, and thus consider the reaction 
\begin{equation} \label{eq:M2:theory:electron_proton_to_hydrogen_photon}
    e^- + p^+ \leftrightarrow H + \gamma.
\end{equation}   
For the photons we will assume $n_\gamma = \equilibriumn[\gamma]$. We adopt the notation of Dodelson \cite{Dodelson}, where $e$ and $p$ denotes free electrons and protons, respectively, while $H$ stands for neutral Hydrogen. 


Instead of computing $\ne$ directly, we will compute the fractional electron density
\begin{equation} \label{eq:M2:theory:X_e_definition}
    \Xe \equiv \frac{\ne}{n_b},
\end{equation}
where $n_b\approx\ne+n_H$ is the total baryon density. Since we neglect Hydrogen, the baryon density can be written as    
\begin{equation} \label{eq:M2:theory:helium_baryon_number_density}
    n_b \approx \frac{\rho_b}{m_H} = \frac{\obn \rhocn}{m_H a^3},
\end{equation}
where $\rhocn\equiv \frac{ H_0^2}{8\pi G}$ is the critical density of the universe today and $m_H$ is the hydrogen mass. Here, we also assume the small difference between the proton mass and Hydrogen mass to be negligible.

At early times, we can assume that the \note{particles} are in equilibrium, and \Eqref{eq:M2:theory:Boltzmann_equation_four_particles_scale_factor} reduces to 
\begin{equation} \label{eq:M2:theory:Boltzmann_equation_recombination_approximation}
    \frac{n_e n_p}{n_H} = \frac{\equilibriumn[e]\equilibriumn[p]}{\equilibriumn[H]}.
\end{equation}
This gives us the Saha equation 
\begin{equation} \label{eq:M2:theory:saha_equation}
    \frac{\Xe^2}{1-\Xe} = \frac{1}{n_b} \pclosed{\frac{m_e \Tb}{2\pi}}^{3/2} e^{-\epsn / \Tb},
\end{equation}
where $\Tb$ is the temperature of the baryons, and $\epsn$ is the Hydrogen ionization energy. The time evolution of $\Tb$ is governed by a differential equation coupled to $\Xe$. However, we will assume that it follows the photon temperature, $T_\gamma$, evolving as  
\begin{equation} \label{eq:M2:theory:baryon_temperature_evolution}
    \Tb = T_\gamma = \tcmb e^{-x}.
\end{equation}
%==
At later times, \Eqref{eq:M2:theory:Boltzmann_equation_recombination_approximation} is no longer a valid approximation, and we have to solve \Eqref{eq:M2:theory:Boltzmann_equation_four_particles_scale_factor}. Taking atomic physics into account, the differential equation we will consider is the Peebles equation  
\begin{equation} \label{eq:M2:theory:Xe_peebles_ODE}
    \dv{\Xe}{x} = \frac{C_r(\Tb)}{H}\bclosed{\beta(\Tb)(1-\Xe) - n_H \alpha^{(2)}(\Tb) \Xe^2 },
\end{equation}
where $C_r(\Tb),\: \beta(\Tb)$ and $\alpha^{(2)}(\Tb)$ are quantities related to interaction effects, and are given by 
\begin{subequations} \label{eq:M2:theory:peebles:ODE_all_individual_terms}
    \begin{align}
        C_r(\Tb) &= \frac{\Lambda_{2s\to1s} + \Lambda_\alpha}{\Lambda_{2s\to1s} + \Lambda_\alpha + \beta^{(2)}(\Tb)}, \label{eq:M2:theory:peebles:C_r} \\
        \Lambda_{2s\to1s} &= 8.227\unit{s^{-1}}, \label{eq:M2:theory:peebles:Lambda_2s_1s} \\
        \Lambda_\alpha &= H \frac{(3\epsn)^3}{(8\pi)^2 n_{1s}}, \label{eq:M2:theory:peebles:Lambda_alpha} \\
        n_{1s} &= (1-\Xe) n_H , \label{eq:M2:theory:peebles:n_1s} \\
        \beta^{(2)}(\Tb) &= \beta(\Tb) e^{3\epsn/4\Tb}, \label{eq:M2:theory:peebles:beta2} \\
        \beta(\Tb) &= \alpha^{(2)}(\Tb) \pclosed{\frac{m_e \Tb}{2\pi}}^{3/2} e^{-\epsn/\Tb}, \label{eq:M2:theory:peebles:beta} \\
        \alpha^{(2)} &= \frac{64\pi}{\sqrt{27\pi}} \frac{\alpha^2}{m_e^2} \sqrt{\frac{\epsn}{\Tb}} \phi_2(\Tb), \label{eq:M2:theory:peebles:alpha_2} \\
        \phi_2(\Tb) &= 0.448\ln(\epsn/\Tb). \label{eq:M2:theory:peebles:phi_2} 
    \end{align}
\end{subequations}
The main reason for these corrections is that Hydrogen production is inefficient, even at $\Tb\simeq\epsn$. A recombination process where the electron goes directly into the Hydrogen ground state is likely to produce a photon with energy higher than $\epsn$, which will ionize another nearby Hydrogen atom, resulting in no net Hydrogen production. Efficient recombination is therefore obtained by a recombination to an excited state, after which the atom can decay into the ground state, which is a slow process. \note{Added the above paragraph. Proofread later, and check physics.}

The initial condition is chosen as the final value of $\Xe$ obtained from the Saha equation, as we further discuss in \secref{ssec:M2:implementations}   

\subsubsection{Sound Horizon \note{(working title)}} \label{sssec:M2:sound_horizon}
Before recombination happens, baryons and photons are tightly coupled. For this reason, the baryons and photons behave as if they were a single fluid, and the sound speed of this fluid is given by \note{(citation)}
\begin{equation} \label{eq:M2:theory:soundspeed_R_baryons_and_photons}
    c_s(x) = \frac{c}{\sqrt{3}} \sqrt{\frac{R(x)}{1+R(x)}},\quad R(x) = \frac{4\og(x)}{3\ob(x)}.
\end{equation}
\note{(Explicit x-dependence?)} The total co-moving distance a sound wave in this plasma could have travelled since the Big Bang is known as the sound-horizon, which is given as 
\begin{equation} \label{eq:M2:theory:sound_horizon_integral}
    s(x)=\int_{-\infty}^x \frac{\dd x' c_s}{\H}.
\end{equation}
From this we can compute the sound horizon at decoupling, $r_s = s(x_\mathrm{rec})$, which is an important quantity of the CMB. Thus, we have an additional ODE to solve, 
\begin{equation} \label{eq:M2:theory:sound_horizon_ODE}
    \dv{s(x)}{x} = \frac{c_s}{\H},
\end{equation}
with $s(x_\mathrm{ini})=\frac{c_s(x_\mathrm{ini})}{\H(x_\mathrm{ini})}$ as the initial condition, following the same reasoning as we did for $\eta'(x)$ in \Eqref{eq:M1:theory:eta_of_xstart_analytical_approximation}.

\note{Relate the sound-horizon to the Power Spectrum as well.}
