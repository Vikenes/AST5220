
\subsection{Implementation details}\label{ssec:M2:implementations} 
To compute $\Xe$, we use the Saha equation initially, as it is a good approximation at early times when $\Xe \approx 1$. This is also the regime where the Peebles equation is unstable, and we therefore consider the Saha equation for $\Xe>\Xe^\mathrm{tol}$. Once we reach $\Xe<\Xe^\mathrm{tol}$ we use the final value from the Saha equation as our initial condition to solve the Peebles equation. The Peebles equation is then used all the way to today, at $x=0$. \note{MENTION THE TOLERANCE VALUE CHOSEN.}

\subsubsection{Solving the Saha Equation}\label{sssec:M2:implementations:solving_saha}
Solving the Saha equation is done by solving a quadratic formula for $\Xe$. At early times, however, the RHS of \Eqref{eq:M2:theory:saha_equation} will be enormous, and may cause numerical errors when solving the quadratic formula. To avoid this, we use the first order approximation $\sqrt{1+x}\approx 1 + \frac{x}{2}$ for $\abs{x}\ll1$ at early times. The Saha equation is thus implemented as   
\begin{equation}
    \Xe = \begin{cases}
        1, \quad & y>10^7, \\
        \frac{y}{2}\bclosed{-1 + \sqrt{1 + 4/y}},\quad & y\leq 10^7,
    \end{cases}
\end{equation}  
where $y$ refers to the RHS of \Eqref{eq:M2:theory:saha_equation}. Since $\Xe$ is strictly positive, we have omitted the negative solution. The exact value of $10^7$ is chosen to ensure both $\Xe \ngtr 1$, and $\Xe\nless\Xe^\mathrm{tol}$ when the quadratic formula is to be used.

\subsubsection{Solving the Peebles Equation}\label{sssec:M2:implementations:solving_peebles}
To solve \Eqref{eq:M2:theory:Xe_peebles_ODE} numerically, we follow the same procedure as we did for $\eta(x)$, but for the initial condition we use the final value of $\Xe$ that we obtained from the Saha equation. 

At late times, when the baryon temperature gets low, the exponent term in \Eqref{eq:M2:theory:peebles:beta2} for $\beta^{(2)}(T_b)$ become sufficiently large to yield an overflow. However, this is also where $\beta(\Tb)\to0$, due to its exponential factor (\Eqref{eq:M2:theory:peebles:beta}). This exponential factor causes $\beta^{(2)}(T_b)\to0$ at late times. To avoid overflow, we implement the equation for $\beta^{(2)}(T_b)$ as 
\begin{equation}
    \beta^{(2)}(T_b) = \begin{cases}
        0,\quad &\epsn/\Tb > 200, \\
        \beta(\Tb) e^{3\epsn/4\Tb},\quad &\epsn/\Tb \leq 200.
    \end{cases}
\end{equation} 

\note{(Mention splining of log's and so forth?)}

\subsubsection{Optical depth and visibility function} \label{sssec:M2:implementations:optical_depth}
Once $\Xe$ is obtained we get $\nex$ from \Eqref{eq:M2:theory:X_e_definition}, and can then solve \Eqref{eq:M2:theory:tau_ODE} for $\tau(x)$ with the aforementioned initial condition of $\tau(x=0)=0$. We therefore integrate backwards, starting from $x=0$. The visibility function is now easily obtained, as $\tau'(x)$ is given analytically by \Eqref{eq:M2:theory:tau_ODE}. From this, we get immediately $\gx$. 

We also need the second derivative of $\tau$, as well as the first two derivatives of $\g$. For $\tau''(x)$, we compute it from numerically differentiating the $\tau'(x)$ data. \footnote{Couldn't we just compute it analytically? I get a tiny "bump" in $\tau''$ near $X_e^\mathrm{tol}$, and I don't know if this will cause problems later on, and therefore didn't want to spend too much time on it if wasn't needed.} We use this to compute 
\begin{equation} \label{eq:M2:implementations:dg_dx}
    \g'(x) = \vclosed{\tau'(x)^2 - \tau''(x)} e^{-\tau(x)}, 
\end{equation}
and obtain $\g''(x)$ by numerically differentiating $\g'(x)$. This is done to avoid potential errors that may occur when numerically computing the second derivative of $\g(x)$, if $\g(x)$ is somewhat ill-behaved at certain times. \footnote{I don't know if this is useful, or helps, at all. I don't know if we care about the derivatives at $x$-values where this might be an issue.} 