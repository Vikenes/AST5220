
\subsection{Implementation details}\label{ssec:M2:implementations} 

In order to compute the optical depth and the visibility we must first compute the electron density. At early times all Hydrogen is ionized, and we therefore have $\xe\approx 1$ \note{(equal to?)}, and $\xe$ can thus be well approximated by the Saha equation in this regime. As $\xe$ decreases, the Saha approximation is no longer valid, and we must resort to the Peebles equation to compute $\xe$. We compute $\xe$ by first solving the Saha equation, and when $\xe<\xe^\mathrm{tol}$ we resort to the Peebles equation, using the preceding value of from the Saha equation as our initial condition.   

\subsubsection{Solving the Saha Equation}\label{sssec:M2:implementations:solving_saha}
Solving the Saha equation is done by solving a quadratic formula for $\xe$. At early times, however, the RHS of \Eqref{eq:M2:theory:saha_equation} will be \note{huge}, and may result in numerical errors when we implement the quadratic formula numerically. To avoid this, we use the first order approximation $\sqrt{1+x}\approx 1 + \frac{x}{2}$ for $\abs{x}\ll1$ at early times. The equation we will solve is thus 
\begin{equation}
    \xe = \begin{cases}
        1, \quad & y>10^7, \\
        \frac{y}{2}\bclosed{-1 + \sqrt{1 + 4/y}},\quad & y\leq 10^7,
    \end{cases}
\end{equation}  
where we have used $y$ as an abbreviation for the RHS of \Eqref{eq:M2:theory:saha_equation}, and omitted the negative solution as $\xe$ is a strictly positive quantity. The exact value of $10^7$ is chosen to ensure $\xe \ngtr 1$. 

\subsubsection{Solving the Peebles Equation}\label{sssec:M2:implementations:solving_peebles}
We will solve the Peebles equation using Saha as the initial condition. To solve \Eqref{eq:M2:theory:Xe_peebles_ODE} numerically, we follow the same procedure as we did for $\eta(x)$, but for the initial condition we use the final value of $\xe$ that we obtained from the Saha equation. Once $\xe$ is obtained we get $\nex$ from \Eqref{eq:M2:theory:X_e_definition}. However, at late times, when the baryon temperature gets low, the exponent term in \Eqref{eq:M2:theory:peebles:beta2} for $\beta^{(2)}(T_b)$ will yield an overflow. However, the exponential factor in $\beta(\tb)\to0$ results in $\beta^{(2)}(T_b)\to0$ at these temperatures. To avoid numerical errors, we therefore compute 
\begin{equation}
    \beta^{(2)}(T_b) = \begin{cases}
        0,\quad &\epsn/\tb > 200, \\
        \beta(\tb) e^{3\epsn/4\tb},\quad &\epsn/\tb \leq 200.
    \end{cases}
\end{equation} 

\subsubsection{Optical depth} \label{sssec:M2:implementations:optical_depth}
With $\nex$ computed, we can solve \Eqref{eq:M2:theory:tau_ODE} for $\tau(x)$ with the aforementioned initial condition of $\tau(x=0)=0$. Its derivative, $\tau'(x)$, is given by the ODE and is thus \note{trivial}. From this, we get immediately $\gx$. For the second derivative of $\tau$, we compute it by \note{numerical differentiation of the splines.}   

