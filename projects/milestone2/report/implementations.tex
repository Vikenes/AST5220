
\subsection{Implementation details}\label{ssec:M2:implementations} 

\subsubsection{Solving the Saha Equation}\label{sssec:M2:implementations:solving_saha}

In order to compute the optical depth and the visibility we must first compute the electron density. At early times all Hydrogen is ionized, and we therefore have $X_e\approx 1$, and $X_e$ can thus be approximated by the Saha equation, which involves solving a quadratic formula for $X_e$. At early times, however, the RHS of Eq. \eqref{eq:M2:theory:saha_equation} will be \note{huge}, and may result in numerical errors when we implement the quadratic formula numerically. To avoid this, we use the first order approximation $\sqrt{1+x}\approx 1 + \frac{x}{2}$ for $\abs{x}\ll1$ at early times. The equation we will solve is thus 
\begin{equation}
    X_e = \begin{cases}
        1, \quad & y>10^7, \\
        \frac{y}{2}\bclosed{-1 + \sqrt{1 + 4/y}},\quad & y\leq 10^7,
    \end{cases}
\end{equation}  
where we have used $y$ as an abbreviation for the RHS of Eq. \eqref{eq:M2:theory:saha_equation}, and omitted the negative solution as $X_e$ is a strictly positive quantity. 
