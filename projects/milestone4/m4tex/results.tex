\subsection{Results}\label{ssec:M4:results}

\subsubsection{Photon multipoles} \label{sssec:M4:results:photons_multipoles}
\begin{figure}[ht!]
    \includefig{integrand_thetas\pspecresolution\ellintegrand}
    \caption{Integrand.}
    \label{fig:M4:results:integrand_thetas}
\end{figure}


\begin{figure}[ht!]
    \includefig{thetas\pspecresolution\ellthetas}
    \caption{Thetas. }
    \label{fig:M4:results:thetas}
\end{figure}

\subsubsection{CMB power spectrum} \label{sssec:M4:results:angular_power_spectrum}
The angular power spectrum is shown in \figref{fig:M4:results:cells}. We also include data from \cite{Planck2020} for low $\ell$ values. At higher $\ell$, neutrinos, helium and reionization becomes important, and the predicted value of $\cl$ would deviate from that of observations \pnote{Write about how/why these things affect it.} We plot the CMB power spectrum   
\begin{figure}[ht!]
    \includefig{cells\pspecresolution}
    \caption{Angular power spectrum.}
    \label{fig:M4:results:cells}
\end{figure}


\begin{figure}[ht!]
    \includefig{cells_components\pspecresolution}
    \caption{Angular power spectrum.}
    \label{fig:M4:results:cells_components}
\end{figure}

\subsubsection{Matter power spectrum} \label{sssec:M4:results:matter_power_spectrum}
\begin{figure}[ht!]
    \includefig{matterPS_nk1000}
    \caption{Matter power spectrum.}
    \label{fig:M4:results:matterPS_nk1000}
\end{figure}